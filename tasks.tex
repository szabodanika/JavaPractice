\documentclass{article}

\usepackage[utf8]{inputenc}
\usepackage[hungarian]{babel}
\usepackage{xcolor}
\usepackage{listings}
\renewcommand{\lstlistingname}{Kodreszlet}%
\lstset{language=Java,
backgroundcolor=\color[HTML]{ebebeb},
keywordstyle={\bfseries \color[HTML]{5d00ff}},
frame=single,
basicstyle=\footnotesize\ttfamily,
captionpos=b,
tabsize=2,
numbers=left,
aboveskip=2em,
belowskip=1em
}
\usepackage{graphicx}

\title{Java SE Gyakorlo Feladatsor 1}
\author{Szabo Daniel}

\date{\today}

\begin{document}

    \maketitle
    \begin{abstract}

    \end{abstract}
    \tableofcontents{}

    \newpage


    \section{Tesztek Futtatasa}

    Ez a feladatsor tesztekkel lett kiegeszitve, ami lehetove teszi szamodra, hogy azonnal visszajelzest kapj, hogy megfeleloen mukodik-e egy bizonyos feladatra irt megoldasod, illetve a megoldas ertekelojenek is segit egy pillanat alatt ellenorizni, hogy mukodokepes megoldast adtal be. A tesztelesi keretrendszer mukodesi elve miatt a feladatoknak muszaj az OOP elveit hasznalnia, a metodusoknak parametereket befogadnia es visszateresi ertekeket adnia. Erdemes a teszteket minden valtoztatas utan lefuttatni, hogy folyamatosan lasd, esetleg elrontottal-e valamit egy valtoztatasoddal, ami korabban mukodott.

    A teszteket a kovetkezo keppen tudod futtatni:

    \paragraph{Eclipse}

    \paragraph{IntelliJ IDEA}


    \section{Szamologep}

    \paragraph{Bevezetes}

    Ebben a feladatban a \lstinline{Szamologep} osztaly fuggvenyeit fogjuk implementalni. Az egyszeruseg kedveert ebben az osztalyban csak \lstinline{float} adattipust hasznalunk szamitasokhoz. A feladat 4 reszbol all, minden reszben egy metodus hianyzo logikajat kell potolnod. Egy pelda metodus \lstinline{osszead()} segitsegkent el lett helyezve a forraskodban.

    \begin{lstlisting}[language=Java, caption=Pelda Metodus]
	public float osszead(float a, float b){
		return a + b;
	}
    \end{lstlisting}

    \subsection{kivon(a,b)}
    A \lstinline{kivon} fuggveny a ket parameterkent kapott szam kulonbseget kell visszaadja, peldaul:\newline
    \lstinline[mathescape]{kivon(4,3) $\rightarrow$ 1}\newline
    \lstinline[mathescape]{kivon(-2.5,3) $\rightarrow$ -5.5}\newline
    \lstinline[mathescape]{kivon(0,2.25) $\rightarrow$ -2.25}\newline

    \subsection{szoroz(a,b)}
    A \lstinline{kivon} fuggveny a ket parameterkent kapott szam szorzatat kell visszaadja, peldaul:\newline
    \lstinline[mathescape]{szoroz(4,3) $\rightarrow$ 12}\newline
    \lstinline[mathescape]{szoroz(-2.5,3) $\rightarrow$ -7.5}\newline
    \lstinline[mathescape]{szoroz(0,2.25) $\rightarrow$ 0}\newline

    \subsection{oszt(a,b)}
    Az \lstinline{oszt} fuggveny a ket parameterkent kapott szam hanyadosat kell visszaadja, peldaul:\newline
    \lstinline[mathescape]{kivon(4,3) $\rightarrow$ 1}\newline
    \lstinline[mathescape]{kivon(-2.5,3) $\rightarrow$ -5.5}\newline
    \lstinline[mathescape]{kivon(0,2.25) $\rightarrow$ -2.25}\newline

    \subsection{hatvany(a,b)}
    A \lstinline{hatvany} fuggveny a ket parameterkent kapott szam hatvanyat kell visszaadja, peldaul:\newline
    \lstinline[mathescape]{hatvany(2,3) $\rightarrow$ 8}\newline
    \lstinline[mathescape]{hatvany(2,-2) $\rightarrow$ 0.5}\newline
    \lstinline[mathescape]{hatvany(0,2.25) $\rightarrow$ 0}\newline

    \subsection{negyzetgyok(a)}
    A \lstinline{negyzetgyok} fuggveny a ket parameterkent kapott szam kulonbseget kell visszaadja,
peldaul:\newline
\lstinline[mathescape]{negyzetgyok(4) $\rightarrow$ 2}\newline
\lstinline[mathescape]{negyzetgyok(1) $\rightarrow$ 1}\newline
\lstinline[mathescape]{negyzetgyok(-2) $\rightarrow$ hiba (-1)}\newline

    \section{Ciklusok}

    \paragraph{Bevezetes}

    \sub{Feladatok}


    \section{Tombok}

    \paragraph{Bevezetes}

    \paragraph{Feladatok}


    \section{Versenyauto}

    \paragraph{Bevezetes}

    \paragraph{Feladatok}

    \paragraph{Feladatok}
\end{document}