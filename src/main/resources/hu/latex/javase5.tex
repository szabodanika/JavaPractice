\documentclass{article}

\usepackage[utf8]{inputenc}
\usepackage[hungarian]{babel}
\usepackage{hyperref}
\usepackage{xcolor}
\usepackage{listings}
\renewcommand{\lstlistingname}{Kodreszlet}%
\lstset{language=Java,
backgroundcolor=\color[HTML]{ebebeb},
keywordstyle={\bfseries \color[HTML]{5d00ff}},
frame=single,
basicstyle=\footnotesize\ttfamily,
captionpos=b,
tabsize=2,
numbers=left,
aboveskip=2em,
belowskip=1em
}
\usepackage{graphicx}
\usepackage{float}
\restylefloat{table}
\let\l\lstinline

\title{%
Java SE 5 \\
\large Oroklodes az Objektum Orientalt Programozasban}

\author{Szabo Daniel\\daniel.szabo99@outlook.com}

\date{\today}

\begin{document}

\maketitle
\begin{abstract}
Ebben a feladatsorban az objektum orientalt programozas egyik legfontosabb eszkozevel, az oroklodessel ismerkedunk meg, ami segitsegevel a programunk flexibilisebb, rendezettebb lesz.
Az oroklodes lehetove teszi szamunkra, hogy meglevo funkcionalitast kibovitsunk, illetve magunk is bovitheto kodot hozzunk letre, ezzel megalapozva a komplex, modularis rendszerekben valo munkat.
\end{abstract}

\newpage

\tableofcontents{}

\newpage

\section{Oroklodes Alapjai}
\paragraph{Bevezetes}

Az oroklodes az objektum orientalt elmeletben azt jelenti, hogy egy osztaly alosztalya lehet egy masik osztalynak. Az alosztaly ugyanazokat a jellemzoket illetve eljarasokat fogja tartalmazni mint a szuloosztaly, azokat orokli. Oroklodes tehat tipus-altipus kozott van, peldaul szamitogep-laptop, jarmu-auto, allat-kutya.

\subsection{Osztalyvaltozok Oroklese}
A Java nyelvben egy osztalynak tobb alosztalya is lehet, de un. tobbszoros oroklodest, (tehat egy osztalybol egyszerre tobb masik osztalyt) nem tamogat. Az osztalyt amibol orokolni szeretnenk (tehat kiboviteni) a \l{extends} kulcsszoval hatarozhatjuk meg:

\begin{lstlisting}[language=Java, caption=A Laptop osztaly kiboviti a Szamitogep osztalyt]
class Szamitogep {
    String gyarto;
    String modell;
    int processzorMagok;
    int memoria; // GB
    int hattertar; // GB
}
...
class Laptop extends Szamitogep {
   int tomeg; // kg
   int akkumulatorKapacitas; // mAh
}
\end{lstlisting}

A fenti peldaban letrehozott \l{Laptop} a \l{Szamitogep} osztalyt orokli, ami azt jelenti, hogy a sajat \l{tomeg} es \l{akkumulatorKapacitas} valtozoi mellett orokolte a \l{Szamitogep} osztaly osszes valtozojat is.

\newpage

\subsection{Metodusok Oroklese}

A kibovitett osztalybol nem csak osztalyvaltozokat, hanem metodusokat is oroklunk, tehat a lenti peldaban, miutan hozaadtunk egy uj \l{bekapcsol()} metodust a \l{Szamitogep}, az automatikusan oroklodik a \l{Laptop} osztalyba, ahol azt majd minden peldanyon meg lehet hivni, ugyan ugy mint ha eleve a Laptop osztalyban lett volna deklaralva.

\begin{lstlisting}[language=Java, caption=A Laptopnak is lesz bekapcsol() fuggvenye]
class Szamitogep {
    ...
    void bekapcsol() {
        System.out.println("Betoltes...");
    }
}
...
class Laptop extends Szamitogep {

}
\end{lstlisting}

\subsection{Tobbszoros Oroklodes}

Bar egy osztaly nem bovithet ki rogton tobb osztaly, lehetseges egy olyan osztaly kiboviteni ami onmaga is egy alosztaly.

\begin{lstlisting}[language=Java, caption=Tobbszintu oroklodes]
class Szamitogep {
    ...
}
...
class Laptop extends Szamitogep {
   ...
}
...
class TwoInOne extends Laptop {
    ...
}
\end{lstlisting}

A fenti peldaban letrehozott \l{Laptop} a \l{Szamitogep} osztalyt orokli, ami azt jelenti, hogy a sajat \l{tomeg} es \l{akkumulatorKapacitas} valtozoi mellett orokolte a \l{Szamitogep} osztaly osszes valtozojat is.

\newpage

\section{Konstruktorok}

Ha a szuloosztaly konstruktoranak vannak parameterei, akkor azokat a parametereket az alosztalyban ki kell elegiteni. Erre a \l{super} kulcsszot hasznaljuk, aminek segitsegevel a szuloosztaly konstruktorat hivjuk meg. A lenti peldaban a \l{Teglalap} osztaly konstruktora ket parametert ker (magassag, szelesseg), igy a \l{Negyzet} alosztaly konstruktoraban kotelezo ezt a ket erteket megadni. Mivel egy negyzet minden oldala egyenlo hosszu lesz, a \l{Negyzet} konstruktoraban csak egy szamot kerunk be, es azt a szamot adjuk meg a \l{Teglalap} konstruktor mindket ertekenek.

\begin{lstlisting}[language=Java, caption=Szuloosztaly konstruktoranak meghivasa]
class Teglalap {
    int a, b;

    Teglalap(int a, int b) {
        this.a = a;
        this.b = b;
    }
}
...
class Negyzet {
    Negyzet(int a) {
        super(a, a);
    }
}
\end{lstlisting}


A kovetkezo peldaban az alosztaly nem kevesebb, hanem tobb erteket ker be mint a szuloosztalya.

\begin{lstlisting}[language=Java, caption=Szuloosztaly konstruktoranak meghivasa]
class Jarmu {
    String modell;
    int ev;

    Jarmu(String modell, int ev) {
        ...
    }
}
...
class Auto {
    String rendszam;
    int ajtok;

    Auto(String modell, int ev, String rendszam, int ajtok) {
        super(gyarto, modell, ev);
        this.rendszam = rendszam;
        this.ajtok = ajtok;
    }
}
\end{lstlisting}

\newpage

\section{Hozzaferes Modositok}

Az elozo feladatsorban mar megismerkedtunk ket hozzaferes modositoval, a \l{private}-tal es a "default"-tal. Mivel mar tudjuk, mit jelent az, ha egy osztalynak egy masik osztaly alosztalya, igy ertelmezhetjuk a masik ket hozzaferes modositot is. A \l{protected} es a \l{public} kulcsszavakkal a valtozokat es metodusokat elerhetove tehetjuk a a csomagon kivulrol is, azonban a \l{protected} a hozzaferest limitalni fogja az osztalyunk alosztalyaira. A {private} elemek tovabbra is az osztalyon belul lesznek csak elerhetok, a csomagon belul levo alosztalyok sem ferhetnek hozza.

\begin{table}[H]
    \begin{tabular}{|l|l|l|l|l|}
        \hline
        \textbf{Kulccszo}  & \textbf{\begin{tabular}[c]{@{}l@{}}Osztalyon \\ belul\end{tabular}} & \textbf{\begin{tabular}[c]{@{}l@{}}Csom. belul\\ barhonnan\end{tabular}} & \textbf{\begin{tabular}[c]{@{}l@{}}Csom. kivul\\ alosztalybol\end{tabular}} & \textbf{\begin{tabular}[c]{@{}l@{}}Csom. kivul\\ barhonnan\end{tabular}} \\ \hline
        \textbf{private}   & \textbf{IGEN}                                                       & NEM                                                                         & NEM                                                                            & NEM                                                                         \\ \hline
        \textbf{(nincs)}          & IGEN                                                                & \textbf{IGEN}                                                               & NEM                                                                            & NEM                                                                         \\ \hline
        \textbf{protected} & IGEN                                                                & IGEN                                                                        & \textbf{IGEN}                                                                  & NEM                                                                         \\ \hline
        \textbf{public}    & IGEN                                                                & IGEN                                                                        & IGEN                                                                           & \textbf{IGEN}                                                               \\ \hline
    \end{tabular}
    \caption{Hozzaferes Modositok}
    \label{tab:hozzaferes-modositok}
\end{table}

\section{Polimorfizmus}

A polimorfizmus az objektum orientalt programozas foelveinek egyike, jelentese pedig "tobbalakusag".
Akkor beszelunk polimorfizmusrol, amikor egy tipus peldanyai tobb alakot is felvehetnek.
Ennek a legkezenfekvobb peldaja az, amikor egy osztaly alosztalyai felulirjak a szuloosztaly metodusait.

\begin{lstlisting}[language=Java, caption=Pelda osztalyok metodus felulirassal]
class A {
    void M() {
        System.out.println("A");
    }
}
...
class B extends A {
    void M() {
        System.out.println("B");
    }
}
...
class C extends A{
    void M() {
        System.out.println("C");
    }
}
\end{lstlisting}

\newpage

Ha \l{A} osztalyunkat kiboviti a \l{B} es \l{C} osztaly, akkor egy \l{A} tipussal deklaralt valtozoba \l{B} es \l{C} tipusokat is elhelyezhetunk.

\begin{lstlisting}[language=Java, caption=Polimorfizmus pelda]
A a1 = new A();
A a2 = new B();
A a3 = new C();
\end{lstlisting}

Ez azt jelenti, hogy bar mind a harom valtozonk A tipusu, mivel mas osztalyok peldanyai, az \l{M()} metodus mind a harom peldanyon mast fog kiirni a konzolra.

\subsection{@Override annotacio}

Metodusok felulirasakor erosen ajanlott az \l{@Override} annotacio hasznalata, amivel egyertelmuen jelezzuk a Java-nak, hogy a szulosztalyban levo metodus viselkedeset fogjuk felulirni, de a ki- es bemeneteit nem.
Ha azokat is felulirjuk, peldaul egy uj parametert adunk hozza az alosztalyban levo metodushoz, akkor az nem fogja felulirni a szuloosztaly metodusat, hanem uj metoduskent adodik hozza az alosztalyhoz.
Egy \l{@Override} annotacioval megjelolt metodusnal ebben az esetben mar futtatas elott figyelmeztetest fogunk kapni.
Ha egy metodust az annotacio nelkul irunk felul, azt kitakarasnak vagy arnyekolasnak hivhatjuk.

\begin{lstlisting}[language=Java, caption=Override annotacio hasznalata]
...
class C extends A{
    @Override
    void M() {
        System.out.println("C");
    }
}
\end{lstlisting}

\newpage

\subsection{Instanceof operator}

Az \l{instanceof} operator segitsegevel ellenorizheted, hogy egy valtozo egy bizonyos alosztalyahoz tartozo objektumot tartalmaz-e. A lenti peldaban az elozo \l{A}, \l{B}, \l{C} osztalyokra hivatkozunk.

\begin{lstlisting}[language=Java, caption=Instanceof hasznalata]
A a = new B();
if(a instanceof B) {
    System.out.println("Ez egy B objektum");
} else if(a instanceof C) {
    System.out.println("Ez egy C objektum");
}
\end{lstlisting}

Ha meggyozodtel rola, hogy az objektumod egy bizonyos alosztalyhoz tartozik, azt biztonsagban konvertalhatod az altipusba, ezzel helyben elerhetove teve az alosztaly sajatos metodusait, valtozoit.

\begin{lstlisting}[language=Java, caption=Instanceof hasznalata]
class A {
    // ures osztaly
}
...
class B extends A{
    void M() {
        System.out.println("En egy B vagyok!");
    }
}
...
A a = new B();

if(a instanceof B) {
    B b = (B) a;
    b.M();
}
\end{lstlisting}

\newpage

\subsection{Final kulccszo}

Ha meg szeretned akadalyozni, hogy az osztalyodnak alosztalyai johessenek letre, vagy egy metodusod felul legyen irva egy alosztalyban, egy osztaly vagy metodust ellathatsz \l{final} kulcsszoval.

\begin{lstlisting}[language=Java, caption=Final osztaly amit nem lehet kiboviteni]
final class Szamologep() {
    void osszead(int a, int b) {
        return a + b;
    }
}
\end{lstlisting}

\begin{lstlisting}[language=Java, caption=Osztaly aminek lehet alosztalya, de az osszead metodusat nem lehet felulirni]
class Szamologep() {
    final void osszead(int a, int b) {
        return a + b;
    }
}
\end{lstlisting}

\subsection{Super kulcsszo}

A \l{super} kulcsszo nem csak konstruktoroknal hasznalatos, altalanossagban az osztaly szuloosztalyara mutat, a \l{this}-nek felel meg.
Segitsegevel meg tudod hivni egy felulirt metodus eredeti verziojat a szuloosztalyban.
Az alabbi peldaban a \l{Zsiraf} osztaly kiboviti a bemutatkozasat azzal, hogy megmondja a magassagat.

\begin{lstlisting}[language=Java, caption=Override annotacio hasznalata]
class Allat {
    String nev;
    int kor;

    String bemutatkozas() {
        return "A nevem " + nev + " es " + kor + " eves vagyok";
    }
}
...
class Zsiraf {
    int magassag;

    @Override
    String bemutatkozas() {
        return super.bemutatkozas() + " A magassagom "
            + magassag + " meter.";
    }
}
\end{lstlisting}

\section{Object osztaly}

Ugyan ez szamunkra nem lathato, minden osztaly automatikusan kiboviti a \l{java.lang.Object} osztalyt. Ez az osztalyhierarchia csucsa, ami meghataroz olyan metodusokat amit minden Java objektumnak kotelezo tartalmaznia, azonban ezeket az alosztalyokkal felulirhatjuk. Figyelmesen tanulmanyozd az \l{Object} osztalyt es keresd meg az osztaly Java dokumentaciojat az interneten.

\subsubsection{toString() metodus}

A \l{toString()} metodus hatarozza meg, hogy egy objektumot hogyan reprezentalunk \l{String} alakban.

Amikor fontos szamunkra, hogy egy osztaly peldanyait emberek szamara is erthetoen jelenitsunk meg, olyankor ezt a metodust irjuk felul es String-kent adjuk vissza az objektumunk adatait, vagy barmit amit a szituacio kivan.
Az alapveto implementacioja ennek a metodusnak csak az objektum osztalyat es hash kodjat fogja visszaadni, peldaul \l{Termek@1fee6fc};

\subsubsection{equals() metodus}

Ez a metodus hatarozza meg, hogy mi alapjan tekintjuk egy osztaly ket peldanyat egyenlonek.
Fontos megjegyezni, hogy ennek a metodusnak a felulirasa eseteben kotelezo a \l{hashCode} metodust is felulirni, hogy az osztalyod konzisztenszen mukodjon hash alapu adatszerkezetekben, peldaul \l{HashMap()}.
Szerencsere ezt egyszeruen megtehetjuk az \l{Objects.hash()} metodus hasznalataval.

\begin{lstlisting}[language=Java, caption=Override annotacio hasznalata]
public class Foglalas {
    String nev;
    int ora;

    @Override
    public boolean equals(Object o) {
        ...
    }

    @Override
    public int hashCode() {
        return Objects.hash(nev, ora);
    }
}
\end{lstlisting}

\newpage

\section{Feladatok}

\subsection{Feladat}

Az alabbi fogalomparok kozott melyek vannak szuloosztaly-alosztaly kapcsolatban?
\begin{itemize}
    \item konyv - ujsag
    \item auto - kormanykerek
    \item ruhadarab - nadrag
    \item egesz szam - 12
    \item szam - egesz szam
    \item kepmegjelenito - LCD kijelzo
    \item evoeszkoz keszlet - villa
    \item alkalmazott - nyari gyakornok
\end{itemize}

\subsection{Feladat}

Mi tortenik, ha
\begin{itemize}
    \item megprobalsz egy osztalybol ket masik osztalyt kiboviteni,
    \item megprobalod a Math osztaly kiboviteni,
    \item egy szulo osztalynak statikus metodusai/valtozoi vannak,
    \item egy alosztalybol nem elegited ki a szuloosztaly konstruktoranak parametereit,
    \item egy tetszoleges T tipusu tombben T kulonbozo alosztalyainak objektumait tesszuk,
    \item egy tetszoleges T tipussal deklaralt valtozonak T szuloosztalyanak objektumat adjuk ertekul,
    \item egy tetszoleges T tipus egyik alosztalyaval deklaralt valtozojanak a T osztaly egy masik alosztalyanak objektumat adjuk ertekul,
    \item egy alosztalyban letrehozott metodus vagy osztalyvaltozo neve megegyezik a szuloosztaly egyik metodusaval,
    \item egy statikus metodust egy nem statikus metodussal arnyekolunk vagy forditva,
    \item ket osztaly egymast boviti ki,
    \item egy \l{final} kulcsszoval megjlelolt osztalyt vagy metodust megprobalsz kiboviteni vagy felulirni,
    \item egy \l{@Override} annotacioval megjelolt metodus neve/parameterei/visszateresi erteke nem egyezik meg a szuloosztaly egyik metodusaval sem?
\end{itemize}

\subsection{Feladat}

Figyelmesen tanulmanyozd es probald ki az \l{Object} osztaly \l{toString()}, \l{equals()} metodusainak alapveto implementaciojat, majd ird felul egy sajat osztalyodban.

\subsection{Feladat}

A kovetkezo feladatokban a lenti \l{Termek} osztallyal fogunk dolgozni.

\begin{lstlisting}[language=Java, caption=Feladat]
class Termek {
    private static int kovetkezoAzonosito = 100000;
    private int azonosito;
    private String megnevezes;
    private int ar;

    public Termek(String megnevezes, int ar) {
        this.azonosito = kovetkezoAzonosito++;
        this.megnevezes = megnevezes;
        this.ar = ar;
    }

    public int getAzonosito() {
        return azonosito;
    }

    public String getMegnevezes() {
        return megnevezes;
    }

    public int getAr() {
        return ar;
    }
}
\end{lstlisting}

\subsubsection{Feladat}

Keszits a \l{Termek} osztalynak egy \l{Laptop} alosztalyt.
Minden laptoprol el szeretnenk tarolni a processzor tipusat es a kepatlo hosszat.
A \l{Laptop} alosztaly a szuloosztalya mintajara keszuljon, a valtozok hozzaferhetosege, az osztaly mutalhatosaga, konstruktora stb.
egyezzen meg vele.

\subsubsection{Feladat}

Keszits a \l{Termek} osztalynak egy \l{Kamera} alosztalyt.
Minden laptoprol el szeretnenk tarolni a felbontast megapixelben.
A \l{Kamera} alosztaly a szuloosztalya mintajara keszuljon, a valtozok hozzaferhetosege, az osztaly mutalhatosaga, konstruktora stb.
egyezzen meg vele.

\subsubsection{Feladat}

Keszits a \l{Termek} osztalyban egy \l{toString} metodust, amely a kovetkezo formatumu String-ben adja vissza a termek adatait:

\begin{lstlisting}[language=Java, caption=Feladat]
100014    BananaBook Pro 12     226000 Ft
\end{lstlisting}

\subsubsection{Feladat}

Egy \l{Termek} tipusu \l{List}-be helyezz vegyesen \l{Laptop} es \l{Kamera} peldanyokat, majd mindet jelenitsd meg egy ciklus segitsegevel.

\begin{lstlisting}[language=Java, caption=Feladat]
Azon.     Megnev.               Ar
100000    BananaBook Pro 12     226000 Ft
100001    BananaBook Pro 15     265000 Ft
100002    Canikon D6000D        120000 Ft
100003    Canikon D6500D        165000 Ft
\end{lstlisting}

\subsubsection{Feladat}

Az \l{instanceof} operatorral motositsd a ciklusod ugy, hogy csak a laptopokat jelenitse meg.

\subsubsection{Feladat}

Az \l{instanceof} operatorral minden termekhez ird oda, hogy melyik kategoriaba tartozik.

\begin{lstlisting}[language=Java, caption=Feladat]
Azon.     Megnev.               Ar          Kat.
100000    BananaBook Pro 12     226000 Ft   Laptop
100001    BananaBook Pro 15     265000 Ft   Laptop
100002    Canikon D6000D        120000 Ft   Kamera
100003    Canikon D6500D        165000 Ft   Kamera
\end{lstlisting}

\newpage

\subsubsection{Feladat}

Ird felul a \l{Laptop} es \l{Kamera} osztalyokban a \l{toString()} metodust ugy, hogy a ciklusod az alabbi eredmenyt produkalja.

\begin{lstlisting}[language=Java, caption=Feladat]
Azon.     Megnev.               Ar          Tul.            Kat.
100000    BananaBook Pro 12     226000 Ft   ADM i6, 30cm    Laptop
100001    BananaBook Pro 15     265000 Ft   ADM i10, 24cm   Laptop
100002    Canikon D6000D        120000 Ft   24 MP           Kamera
100003    Canikon D6500D        165000 Ft   36 MP           Kamera
\end{lstlisting}


\subsubsection{Feladat}

Modositsd ugy a \l{Termek} osztaly, hogy a \l{equals()} metodussal ket peldanyat osszehasonlitval akkor tekintunk egyenlonek ket termeket, ha az azonositoik megegyeznek.

\end{document}

