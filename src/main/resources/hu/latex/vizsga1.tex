\documentclass{article}

\usepackage[utf8]{inputenc}
\usepackage[hungarian]{babel}
\usepackage{hyperref}
\usepackage{xcolor}
\usepackage{listings}
\renewcommand{\lstlistingname}{Kodreszlet}%
\lstset{language=Java,
backgroundcolor=\color[HTML]{ebebeb},
keywordstyle={\bfseries \color[HTML]{5d00ff}},
frame=single,
basicstyle=\footnotesize\ttfamily,
captionpos=b,
tabsize=2,
numbers=left,
aboveskip=2em,
belowskip=1em
}
\usepackage{graphicx}
\usepackage{float}
\restylefloat{table}

\title{%
Java SE Vizsga 1\\
\large Java Alapok, OOP\\
\large Rendelkezesre allo ido: 90 perc}

\author{Szabo Daniel\\daniel.szabo99@outlook.com}

\date{\today}

\begin{document}

\maketitle

\section{Utasitasok}

\subsection{Projekt megnyitasa}

A vizsgahoz a kovetkezo fajlokra van szukseged:
\begin{itemize}
    \item \lstinline{Aruhaz.java}
    \item \lstinline{Termek.java}
\end{itemize}
Ezeket egy uj Eclipse Java projekt-ben helyezzuk el. A projekt neve legyen \lstinline{JavaSEVizsga1_KeresztnevVezeteknev}. A projekt-ben hozzunk letre egy \lstinline{vizsga} csomagot es abban helyezzuk el a ket osztalyt. Ne felejtsd el a csomag megnevezeset az osztalyokban is modositani.

\subsection{Feladatok megoldasa}

\begin{itemize}
    \item A megoldas soran kovessuk az objektum orientalt programozas iranyelveit, torekedjuk osztalyokat es metodusokat hasznalni a megoldashoz.
    \item Nem mukodo kodot illetve felesleges kommenteket ne hagyjunk a projektben
\end{itemize}

\subsection{Megoldas beadasa}

A teljes projektet egy zip allomanykent mentsuk el. Ezt megtehetjuk a File $\rightarrow$ Export $\rightarrow$ Archive File menuponttal. A tomoritett allomany neve legyen a projekt nevevel azonos. Beadas elott csomagold ki a projektet es ellenorizd, hogy mukodik a megoldasod, majd add be a zip allomanyt.

\newpage

\section{Feladatok}

Mielott hozzakezdenel a feladatok megoldasahoz, figyelmesen olvasd el a kodot amit kaptal es ertelmezd az osztalyok elemeit, keresd meg a hianyzo reszeket es a feladatok helyeit.

\subsection{Feladat}
Egeszitsd ki a \lstinline{Termek} osztalyt a kovetkezokkel:
\begin{itemize}
    \item privat osztalyvaltozok a peldaadat alapjan megadott tipusokkal
    \item komplett konstruktor ami az osztaly minden valtozojat beallitja
    \item get metodus minden osztalyvaltozohoz
    \item toString metodus ami visszaad egy egysoros String-et ami tartalmazza a termek minden adatat (lasd: 1. kodreszlet)
\end{itemize}

\begin{lstlisting}[language=Java, caption=Pelda Termek.toString() eredmeny]
Alma    Elelmiszer  50 Ft
\end{lstlisting}

\subsection{Feladat}
Az \lstinline{Aruhaz} osztaly \lstinline{megjelenit()} metodusat egeszitsd ki ugy, hogy a termekek listan levo minden termeket jelenitsen meg a konzolban, oszlopnevekkel egyutt.

\begin{lstlisting}[language=Java, caption=Pelda Aruhaz.megjelenit() kimenet]
Elnevezes   Kategoria   Ar
==========================
Alma    Elelmiszer  50 Ft
Banan   Elelmiszer  40 Ft
Korte   Elelmiszer  45 Ft
\end{lstlisting}

\subsection{Feladat}

Az \lstinline{Aruhaz} osztaly \lstinline{termekekSzama()} metodusat egeszitsd ki ugy, hogy az elerheto termekek szamat irja ki a konzolra (ha valtozik a termekek szama, automatikusan valtozzon a kiirt ertek is, ne mindig 15-ot irjon ki).

\begin{lstlisting}[language=Java, caption=Pelda Aruhaz.termekekSzama() kimenet]
Termekek szama: 15db
\end{lstlisting}

\subsection{Feladat}

Az \lstinline{Aruhaz} osztaly \lstinline{elektronikaiTermekekSzama()} metodusat egeszitsd ki ugy, hogy az elerheto elektronikai termekek szamat irja ki a konzolra (ha valtozik a termekek szama, automatikusan valtozzon a kiirt ertek is, ne mindig 8-at irjon ki).

\begin{lstlisting}[language=Java, caption=Pelda Aruhaz.elektronikaiTermekekSzama() kimenet]
Elektronikai termekek szama: 8db
\end{lstlisting}

\subsection{Feladat}

Az \lstinline{Aruhaz} osztaly \lstinline{legdragabbTermek()} metodusat egeszitsd ki ugy, hogy a legdragabb termek adatait jelenitse meg a konzolon.

\begin{lstlisting}[language=Java, caption=Pelda Aruhaz.legdragabbTermek() kimenet]
**** Legdragabb termek ****
Elnevezes   Kategoria   Ar
==========================
Laptop  Elektronika    43920 Ft
\end{lstlisting}

\subsection{Feladat}

Az \lstinline{Aruhaz} osztaly \lstinline{elelmiszerekAtlagAra()} metodusat egeszitsd ki ugy, hogy az elelmiszer kategoriaban levo termekek atlagos arat jelenitse meg.

\begin{lstlisting}[language=Java, caption=Pelda Aruhaz.legdragabbTermek() kimenet]
Elelmiszerek atlagara: 50.714287 Ft
\end{lstlisting}


\end{document}