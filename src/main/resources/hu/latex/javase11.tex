\documentclass{article}

\usepackage[utf8]{inputenc}
\usepackage[hungarian]{babel}
\usepackage{hyperref}
\usepackage{xcolor}
\usepackage{listings}
\renewcommand{\lstlistingname}{Kodreszlet}%
\lstset{language=Java,
	backgroundcolor=\color[HTML]{ebebeb},
	keywordstyle={\bfseries \color[HTML]{5d00ff}},
	frame=single,
	basicstyle=\footnotesize\ttfamily,
	captionpos=b,
	tabsize=2,
	numbers=left,
	aboveskip=2em,
	belowskip=1em
}
\usepackage{graphicx}
\usepackage{float}
\restylefloat{table}
\let\l\lstinline

\title{%
Java SE 11 \\
\large Többszálas Programozás}

\author{Szabo Daniel\\daniel.szabo99@outlook.com}

\date{\today}

\begin{document}

	\maketitle
	\begin{abstract}
		Ebben a feladatsorban megismerkedünk a többszálas programozás alapjaival, előnyeivel és néhány felhasználási módjával.
	\end{abstract}

	\newpage

	\tableofcontents{}

	\newpage

	\section{Szálak Létrehozása}
	\paragraph{Bevezetes}

	Egy háttérszál indításához a \l{Thread} osztályt kell példányosítanunk és a \l{.start()} metódussal elindítanunk. Minden szálnak egy \l{Runnable} példányra van szüksége a konstruktorban, ami egy funkcionális interfész, tehát lambda kifejezéssel is implementálhatjuk. Opcionálisan adhatunk egy nevet is a szálunknak, ez a jovőben megkönnyíti a hibakeresést.

	\begin{lstlisting}[language=Java, caption=Új szál indítása]
Thread szal = new Thread(() -> {
	System.out.println("Betoltes elindult.");
	Adatbazis.betolt("adat.txt");
	System.out.println("Betoltes elkeszult.");
}, "BetoltoSzal");

szal.start();
	\end{lstlisting}

	Ha több szál ugyanazon szeretnénk, hogy dolgozzon, implementálhatjuk a \l{Runnable} interfészt egy változóban vagy egy egyedi osztályban.

\begin{lstlisting}[language=Java, caption=Új Runnable készítése]
public class BetoltoFeladat implements Runnable {
	private String file;

	public BetoltoFeladat(String file) {
		this.file = file;
	}

	@Override
	public void run() {
		System.out.println("Betoltes elindult.");
		Adatbazis.betolt(file);
		System.out.println("Betoltes elkeszult.");
	}
}
	\end{lstlisting}

\begin{lstlisting}[language=Java, caption=Új szálak indítása ugyanazzal a Runnable obkaketummal]
BetoltoFeladat bf = new BetoltoFeladat("adat.txt");
Thread szal1 = new Thread(bf, "BetoltoSzal1");
Thread szal2 = new Thread(bf, "BetoltoSzal2");
szal1.start();
szal2.start();
\end{lstlisting}

	\newpage

Az létrehozott szálak a fő programszáltól ("main") külön, a háttérben fognak futni, amint elindítottuk őket, majd leállnak amint a \l{.run()} metódusuk teljesen lefutott. Ez lehetővé teszi számunkra, hogy akár két vagy több folyamat egyszerre történhessen anélkül, hogy az egyik folyamat a másikra várna. Ez alapkövetelménye minden felhasználói felületnek például, ahol a szoftvernek reszponzívnak kell maradnia akkor is, ha a háttérben történő folyamatok megakadnak (pl. megszakadt internetkapcsolat vagy lassú, több másodperces fájloperációk során).

Egy másik jelentős felhasználási formája a szálaknak a feladatok felgyorsítása. Gyakran egy folyamatot felbonhatunk kisebb részekre és azokat eloszthatjuk szálak között, amiket az operációs rendszerünk igazságosan kiosztja a processzorunk magjainak, ezzel akár töredékére csökkentve egy számításigényes folyamat futásidejét.

	\section{Szálak Kezelése}

	\subsection{Szál Elérése}
Helyben indított szálakat a tartalmazó változóikra hivatkozva elérhetünk. A szálakon belül futtatott metódusokban a \l{Thread.currentThread()} statikusan elérhetővé teszi számunkra azt a szálat, amiben a kód jelenleg éppen fut.

\begin{lstlisting}[language=Java, caption=Jelenlegi szál nevének kiírása]
System.out.println("Ezt a parancsot a következő szál futtatja: "
		+ Thread.currentThread().getName());
\end{lstlisting}

	\newpage

	\subsection{Szálak Várakoztatása}

Egy szálat alvó módba helyezhetünk egy milliszekundumban megadott időtartamra a statikus
	\l{Thread.sleep()} metódussal. Ezt használhatjuk időzítésre, animálásra, vagy egy ciklus lelassítására. Az alábbi példában egy while ciklusban ismételten ellenőrizzük, hogy az adat sikeresen be lett-e töltve a lemezről minden másodpercben egyszer. Ha nem altatnánk a szálat ciklusok között, olyan frekvenciával ellenőrizné a kondíciót a programunk, amit csak a processzor lehetővé tesz, ezzel felesleges és veszélyesen sok munkát adva neki, ami a rendszer lelassulását vagy akár teljes leállását okozhatja.

	\begin{lstlisting}[language=Java, caption=Szál altatása ciklusok között]
while(!adat.betoltve()) {
    System.out.println("Betöltés...");
 	try {
		Thread.sleep(1000);
	} catch (InterruptedException e) {
		e.printStackTrace();
	}
}
System.out.println("Adat sikeresen betöltve.");
	\end{lstlisting}

	\subsection{Szálak Megszakítása}

Ahogy az előző példában láttad, a \l{Thread.sleep()} metódus \l{InterruptedException}-t dobhat, ami egy ellenőrzött kivétel. Ez akkor fordulhat elő, ha a szálat megkérjük, hogy álljon le az \l{.interrupt()} metódussal. Nagyon fontos megjegyezni, hogy ez a metódus nem fogja azonnal leállítani a szálat, ha ez ígz lenne, nagyon gyakori lenne, hogy egy szál leállításakor félkész munkát hagy hátra a szál. Ehelyett nekünk kell implementálni, hogy miképp reagáljon erre a kérelemre a folyamatunk. Ha a szál alszik, akkor a try-catch blokkunkkal tehetjük ezt meg, ha pedig aktívan dolgozik, a \l{isInterrupted()} metódussal ellenőrizhetjük, hogy érkezett-e leállítási kérelem, és ennek megfelelően eldönhetjük, hogy mi történjen.

Az alábbi példában a while ciklusunk addig ismétlődik, amíg a futtató szál nem kap megszakítási kérelmet.

\begin{lstlisting}[language=Java, caption=Szál futtatása megszakításig]
while(!Thread.interrupted()) {
	System.out.println("Még mindig futok");
	feladat();
}
\end{lstlisting}

	\newpage

A következő példában pedig folyamatosan alszik a szál (10000 milliszekundumos részletekben), ha pedig alvás közben valaki félbeszakítja, visszatér.

\begin{lstlisting}[language=Java, caption=Szál futtatása megszakításig]
System.out.println("Megyek aludni");
while(true) {
	System.out.println("zzzZZZ");
	try {
		Thread.sleep(10000);
	} catch (InterruptedException e) {
		System.out.println("Ki szakította félbe az álmom?");
		return;
	}
}
\end{lstlisting}

Abban az esetben, amikor egy szál meg kell várjon egy vagy több másik szálat, a főszál futását blokkolhatjuk. Ezt a szál \l{.join()} metódusával tehetjük meg.

Az alábbi példában a main szálból indítunk egy alszálat, majd megvárjuk, hogy a szál lefusson.

\begin{lstlisting}[language=Java, caption=Szálak összekapcsolása]
	szal.start();
	try {
		szal.join();
	} catch (InterruptedException e) {
		e.printStackTrace();
	}
\end{lstlisting}

	\newpage

\section{ExecutorService}

Az ExecutorService használatával rábízhatjuk a Java-ra a szálak kezelését és koncentrálhatunk helyette az elvégzendő feladatok implementására. Az ExecutorService létrehozza és lefuttatja a szálakat helyettünk, illetve számos különféle eljárást is kínál a feladatok elvégzésére. Ezt a rendszert kifejezetten olyan esetekben érdemes használni, amikor nem egy nagyobb feladatot kell elvégezni, hanem sok kisebb, különálló feladatot.

A \l{Runnable} típus mellett \l{Callable} objektumokat is használhatunk, ami egy generikus funkcionális interfész tetszőleges visszatérési értékkel. Visszatérési értékként az ExecutorService egy, a megadott \l{Callable} objektum típusával megegyező típusú \l{Future} objektumot ad eredményként, ami a tartalmazni fogja az eljárásunk visszatérési értékét.

Egy \l{Callable} vagy \l{Runnable} objektumot a \l{.submit()} paranccsal adhatunk hozzá az ExecutorService-hez, ami sorbaállítja a többi elvégzendő feladat mögé, vagy ha nincs más elvégzendő feladat, azonnal lefuttatja.

Lehetőségünk van egyszerre több feladatot is lefuttatni, ha a feladatokat egy listában adjuk át a \l{.invokeAll()} metódusnak, ami eredményként egy \l{Future} típusú List objektummal fog visszatérni, ami minden \l{Callable} eredményét tartalmazza, vagy \l{Runnable} esetében mind üres lesz.

\subsection{SingleThreadExecutor}

	A legegyszerűbb ExecutorService típus egyetlen szálat indít el és azon a szálon fogja az összes sorbaállított feladatot lefuttatni. Az alábbi példában a lambda kifejezéssel létrehozott \l{Callable} 5 másodperc várakozás után 1 értékkel tér vissza. Ezt a feladatot egy egyszálas ExecutorService végzi el, amíg a főszálon másodpercenként kiírjuk, hogy "Betöltés...". Amint a \l{Future} objektumunk készen áll, kiírjuk a konzolra a kapott értéket.

\begin{lstlisting}[language=Java, caption=SingleThreadExecutor és Future használata]
ExecutorService es = Executors.newSingleThreadExecutor();

Future<Integer> f = es.submit(() -> {
	Thread.sleep(5000);
	return 1;
});

while(!f.isDone()) {
	Thread.sleep(1000);
	System.out.println("Betöltés...");
}
System.out.println("Kész\nEredmények:");
try {
	System.out.println(f.get());
} catch (ExecutionException e) {
	e.printStackTrace();
}
\end{lstlisting}

\subsection{FixedThreadPool}

A FixedThreadPool megadott számú szálat fog létrehozni, majd a feladatokat szétosztja a szálak között. Ha az egyik szál elvégezte a feladatát, újabb feladatot kap, amíg el nem fogynak a feladatok. Az alábbi példában egy listát töltünk fel 100 feladattal, amit \l{.submit()} helyett a \l{.invokeAll()} metódussal futtatunk le. Ez a metódus csak akkor fog visszatérni, amikor minden feladat elkészült, így nincs szükségünk külön várakoztatni a főszálat.

\begin{lstlisting}[language=Java, caption=FixedThreadExecutor és Future használata]
List<Callable<Integer>> szamolasok = new ArrayList<>();
for (int i = 0; i < 100; i++) {
	szamolasok.add(() -> {
		Thread.sleep(1000);
		return (int) (Math.random() * 100);
	});
}

ExecutorService es = Executors.newFixedThreadPool(12);

List<Future<Integer>> futureList = es.invokeAll(szamolasok);

System.out.println("Kész\nEredmények:");
for (Future<Integer> future : futureList) {
	try {
		System.out.println(future.get());
	} catch (ExecutionException e) {
		e.printStackTrace();
	}
}
\end{lstlisting}

\subsection{CachedThreadPool}

A CachedThreadPool a FixedThreadPool-hoz hasonlóan több szálon fogja futtatni a feladatokat, de ahelyett, hogy mi szabnánk meg hány szálat készítsen, a feladatok illetve az elérhető processzormagok számától függően dinamikusan változtatja a használt szálak mennyiségét.

\newpage

\section{Szinkronizáció}

Általánosságban nem javasolt több szálról ugyanazzal a változóval dolgozni, de ha mégis szükségünk van erre, meg kell győződnünk arról, hogy amikor a két szál ugyanabban a pillanatban próbál műveletet végezni a változón, mindkét művelet végre lesz hajtva. Vegyük például az alábbi kódrészletet, amiben egy számot növelünk és csökkentünk két külön szálról ugyanabban a pillanatban másodpercenként.

\begin{lstlisting}[language=Java, caption=Változó módosítása két szálról]
new Thread(() -> {
	while(true) {
		System.out.println("ertek + 1 = " + ++ertek);
		try {
			Thread.sleep(1000);
		} catch (InterruptedException e) {}
	}
}).start();
new Thread(() -> {
	while(true) {
		System.out.println("ertek - 1 = " + --ertek);
		try {
			Thread.sleep(1000);
		} catch (InterruptedException e) {}
	}
}).start();
\end{lstlisting}

Azt várnánk, hogy a változó értéke minden másodpercben nő és azonnal visszacsökkent az eredeti értékére, azonban nem ez történik. Amikor a két művelet egyszerre történik, van, hogy az egyik nem fog megtörténni, így véletlenszerű eredménye lehet a programunknak.

\begin{lstlisting}[language=Java, caption=Kimenet]
...
ertek + 1 = 0
ertek - 1 = -1
ertek + 1 = 1
ertek - 1 = 0
ertek + 1 = 2
ertek + 1 = 3
ertek - 1 = 2
ertek + 1 = 3
...
\end{lstlisting}

Ezt úgy kerülhetjük el, hogy a kódunkat szálbiztossá tesszük. Ennek több eszköze is van, az egyik az, hogy egy szálbiztos változótípust használunk.

Primitív \l{int} helyett például használhatunk \l{AtomicInteger}-t:

\begin{lstlisting}[language=Java, caption=AtomicInteger használata]
	new Thread(() -> {
		while(true) {
			System.out.println("ertek + 1 = " + ertek.incrementAndGet());
			try {
				Thread.sleep(1000);
			} catch (InterruptedException e) {}
		}
	}).start();
	new Thread(() -> {
		while(true) {
			System.out.println("ertek - 1 = " + ertek.decrementAndGet());
			try {
				Thread.sleep(1000);
			} catch (InterruptedException e) {}
		}
	}).start();
\end{lstlisting}

Egy másik módszer egy objektum lock-ként való használata. Ilyenkor egy objektumon keresztül kommunikál a két szál, és ha az egyik szál éppen dolgozik az adott változón, a másik szál megvárja, amíg végez:

\begin{lstlisting}[language=Java, caption=Lock használata]
 private static int ertek;
    private static final Object lock = new Object();

    public static void main(String[] args) {
        new Thread(() -> {
            while (true) {
                synchronized (lock) {
                    System.out.println("ertek + 1 = " + ++ertek);
                }
                try {
                    Thread.sleep(1000);
                } catch (InterruptedException e) {
                }

            }
        }).start();
        new Thread(() -> {
            while (true) {
                synchronized (lock) {
                    System.out.println("ertek - 1 = " + --ertek);
                }
                try {
                    Thread.sleep(1000);
                } catch (InterruptedException e) {
                }

            }
        }).start();
    }
\end{lstlisting}
\newpage

\section{Feladatok}

\subsection{Feladat}

Készíts egy programot, ami minden tizedik másodpercben kiírja jelenlegi időt a konzolra. A kimenet a lenti példához hasonlítson. A számlálást bármilyen tetszőleges időponttól kezdheti.

\begin{lstlisting}[language=Java, caption=Kimenet]
11:50
12:00
12:10
12:20
\end{lstlisting}

\subsection{Feladat}

Egészítsd ki az előző feladatot úgy, hogy ha a felhasználó beír egy "q" karaktert és lenyomja az enter gombot, jelenjen meg egy "Kilépés..." üzenet majd lépjen ki az alkalmazás. Kilépésre legyen lehetőség két számolás között és azonnali legyen.

\begin{lstlisting}[language=Java, caption=Kimenet]
11:50
12:00
12:10
12:20
Kilepes...
\end{lstlisting}

\subsection{Feladat}

\end{document}
