\documentclass{article}

\usepackage[utf8]{inputenc}
\usepackage[hungarian]{babel}
\usepackage{hyperref}
\usepackage{xcolor}
\usepackage{listings}
\renewcommand{\lstlistingname}{Kodreszlet}%
\lstset{language=Java,
backgroundcolor=\color[HTML]{ebebeb},
keywordstyle={\bfseries \color[HTML]{5d00ff}},
frame=single,
basicstyle=\footnotesize\ttfamily,
captionpos=b,
tabsize=2,
numbers=left,
aboveskip=2em,
belowskip=1em
}
\usepackage{graphicx}
\usepackage{float}
\restylefloat{table}
\let\l\lstinline

\title{%
Java SE 3 \\
\large Objektum Orientalt Programozas Alapjai}

\author{Szabo Daniel\\daniel.szabo99@outlook.com}

\date{\today}

\begin{document}

\maketitle
\begin{abstract}
Ebben a feladatsorban az objektum orientalt programozas (OOP) alapjaival ismerkedunk meg: az osztalyokkal,
objektumokkal, metodusokkal, osztalyvaltozokkal es az ezek kozotti kapcsolatokkal, lehetosegekkel. Az OOP nem csak a
Java, hanem szamos mas programozasi nyelv fontos resze, ezert nagyon fontos, hogy az OOP alapfogalmait hianytalanul
elsajatitsd.
\end{abstract}

\newpage

\tableofcontents{}

\newpage

\section{Bevezetes az Osztalyokba}

\paragraph{Bevezetes}

Az osztaly egy olyan szerkezet ami leirja egy valos targy, esemeny, koncepcio stb. jellemzoit es egyedi metodusait
(fuggvenyeit). Az osztaly csak egy minta, onmaga nem fog adatot tartalmazni, hanem majd peldanyositaskor hozunk
letre a deklaralt osztaly alapjan egy valtozot ami rendelkezni fog az osztalyban meghatarozott jellemzokkel es metodusokkal.

\subsection{Osztaly deklaralasa}

Egy osztaly hasznalatanak az elso lepese mindig az osztaly deklaralasa. Jellemzoen minden osztaly sajat .java fajlba
kerul, aminek a neve megegyezik az osztaly nevevel. A lenti Kutya osztaly peldaul egy Kutya.java nevu fajlba kerulne.

Ebben az elso peldaban kutyakat szeretnenk modellezni. Minden kutyanak eltaroljuk a nevet, korat es azt, hogy be
lett-e oltva. Minden valtozot kulon deklaralok megfelelo adattipussal. Mivel minden kutyanak mas lesz a neve, kora es
oltottsaga, ezert nem allitok be egyiknek sem kezdoerteket. Amikor en nem allitok be manualisan kezdoerteket egy
ilyen valtozonak, olyankor a tipus alaperteket fogja felvenni (lasd: Java SE 1, 1. tablazat).

\begin{lstlisting}[language=Java, caption=Pelda osztaly]
class Kutya {
    String nev;
    int kor;
    boolean oltott;
}
\end{lstlisting}

Ebben a kovetkezo peldaban egy rendelesi rendszerben egy rendelest irok le. Minden rendelesnek van egy cikkszama, ami
mivel betuket is tartalmazhat, String tipusu lett. Jellemzoen(de nem mindig) a vasarlok 1db-ot rendelnek egy termekbol,
ezert a darabszamnak rogton 1 erteket adok.

\begin{lstlisting}[language=Java, caption=Pelda osztaly]
class Rendeles {
    String cikkszam;
    int darabszam = 1;
    String vasarloNeve;
}
\end{lstlisting}

\newpage

Ezeket a valtozokat osztalyvaltozoknak nevezzuk, es a tipusuk barmi lehet, akar tomb, akar egy egyedi adattipus, vagy
akar egy egyedi adattipusokat tarolo tomb is.

\begin{lstlisting}[language=Java, caption=Pelda osztaly]
class Tanulo {
    String nev;
    int szuletesiEv;
}
...
class Osztaly {
    String osztalyNev;
    Tanulo[] nevsor;
}
\end{lstlisting}

\newpage

\subsection{Peldanyositas}

Miutan meghataroztunk egy uj osztalyt, peldanyositassal tudjuk azt felhasznalni. A peldanyositas tehat az a folyamat
amivel egy osztaly alapjan uj valtozot (objektumot) hozunk letre. A sajat osztalybol valo valtozo letrehozasnak a
szabalyai ugyan azok mint ami barmilyen mas tipusra vonatkozik.

\begin{lstlisting}[language=Java, caption=Webaruhaz termekeit leiro osztaly deklaralasa]
class Termek {
    int azonositoSzam;
    String megnevezes;
    int ar;
}
\end{lstlisting}

A fent definialt Termek osztalybol fogunk most peldanyokat letrehozni, osszesen 2 darabot. Mind a harom valtozo
felvett alapertekeket, de en ezeket szeretnem felulirni az igazi ertekekkel amiket hasznalni fogunk a programban.
Egy objektum jellemzoit a \l{peldany.jellemzo} szintaxissal erem el, es a jellemzovel minden olyan muveletet
vegezhetek amit akkor is vegezhetnek vele, ha nem egy osztaly valtozojakent lenne jelen, tehat akadaly nelkul tudom
kiolvasni, beallitani es modositani.

\begin{lstlisting}[language=Java, caption=Termekek valtozoinak beallitasa]
Termek t1 = new Termek();
Termek t2 = new Termek();

t1.azonositoSzam = 2021001;
t2.azonositoSzam = 2021002;
t1.megnevezes = "Eper";
t2.megnevezes = "Csokolade";
t1.ar = 225;
t2.ar = 360;
\end{lstlisting}

\begin{lstlisting}[language=Java, caption=Termekek atlagaranak kiszamolasa]
int osszeg = t1.ar + t2.ar;
double atlag = osszeg / 2d;
System.out.println("Termekek atlagara: " + atlag + "Ft");
\end{lstlisting}

\newpage

\subsection{Metodusok}

Ebben a peldaban egy szamologep osztalyt hozunk letre aminek ket metodusa van: osszead() es kivon(). Mind a ket
metodus 2-2 parametert fogad be bemenetkent es egy eredmenyt ad.

\begin{lstlisting}[language=Java, caption=Szamologep osztaly metodusokkal]
class Szamologep {
    int osszead(int a, int b) {
        int osszeg = a + b;
        return osszeg;
    }

    int kivon(int a, int b) {
        return a - b;
    }
}
\end{lstlisting}

Egy metodus deklaraciojanak reszeit/lepeseit figyeld meg:
\begin{itemize}
    \item \l{int} a visszateresi tipusa a metodusnak: milyen tipusu valtozot ad eredmenyul,
    \item \l{osszead} a metodus neve, amire majd \l{peldany.metodus()} szintaxissal tudok hivatkozni
    \item \l{(int a, int b)} a metodus bemenetei: azoknak tipusai es helyi elnevezesei
    \item 3. sor: a metodus testenek elso sora, ami a metodus hivasakor fog futni.
    \item 4. sor: a \l{return} kulcsszo hatarozza meg a fuggveny eredmenyet
\end{itemize}

A villanykapcsolo peldajaban mar egy olyan osztalyt hozunk letre aminek osztalyvaltozoi es metodusai is vannak: ez a
leggyakoribb, olyan osztalyokra amik csak adatot tarolnak vagy csak metodusokat deklaralnak, viszonylag ritkan van
szuksegunk.

\begin{lstlisting}[language=Java, caption=Villanykapcsolo osztaly]
class Villanykapcsolo {
    boolean felkapcsolva = false;

    void felkapcsol() {
        felkapcsolva = true;
    }

    void lekapcsol() {
        felkapcsolva = false;
    }
}
\end{lstlisting}

Figyeld meg ebben a peldaban azt is, hogy a metodusoknak nincs bemenete, se kimenete. Mivel nincs
kimenetuk, a visszateresi tipusuk \l{void}-ra van allitva. Amig a nem-\l{void} metodusok minden
esetben felvesznek valami erteket (meg akkor is, ha esetleg csak \l{null}-t), addig a \lstlinline{void}
metodusok semmilyen erteket nem vesznek fel.

\begin{lstlisting}[language=Java, caption=Ez a kod hibat okoz]
Villanykapcsolo nappali = new Villanykapcsolo();
int i = nappali.felkapcsol();
\end{lstlisting}

\begin{lstlisting}[language=Java, caption=Ez a kod nem okoz hibat]
Szamologep sz = new Szamologep();
int eredmeny = sz.osszead(10, 20);
\end{lstlisting}

\newpage

\subsection{Feladatok}

\subsubsection{Feladat}

Mi tortenik, ha
\begin{itemize}
    \item ket azonos nevu osztalyt hozok letre,
    \item egy osztalyban se osztalyvaltozok, se metodusok nincsenek,
    \item egy osztalyvaltozo neve megegyezik egy metodusban levo valtozoval,
    \item egy \l{void} metodusbol ertekkel probalok visszaterni,
    \item egy nem-\l{void} metodusbol nem terek vissza ertekkel
    \item egy metodusbol egy masik metodust meghivok
    \item egy metodus meghivja onmagat
    \item deklaralok egy valtozot a sajat osztalyombol de nem inicializalom, majd egy fuggvenyet megprobalom meghivni
\end{itemize}

\subsubsection{Feladat}

Mit csinal es mit ad eredmenyul a kovetkezo metodus?

\begin{lstlisting}[language=Java, caption=Feladat]
int metodus(int a, int b, boolean c) {
    int x;
    if(c) {
        x = a + b;
    } else {
        x = a - b;
    }
    return x;
}
\end{lstlisting}

\subsubsection{Feladat}

Mit csinal es mit ad eredmenyul a kovetkezo metodus?

\begin{lstlisting}[language=Java, caption=Feladat]
String metodus(String s, String s2) {
    return s + s2;
}
\end{lstlisting}

\subsubsection{Feladat}

Mit csinal es mit ad eredmenyul a kovetkezo metodus?

\begin{lstlisting}[language=Java, caption=Feladat]
void metodus(boolean a, boolean b) {
    if(a) {
        return;
    }

    System.out.println(b);
}
\end{lstlisting}


\subsubsection{Feladat}

Hozz letre a kovetkezo osztalyt:

Nev:

\l{Ora}

Jellemzok:
\begin{itemize}
    \item \l{ora} (egesz szam)
    \item \l{perc} (egesz szam)
\end{itemize}

Metodusok:
\begin{itemize}
    \item \l{printIdo} Kiirja a konzolra az ora jelenlegi idejet a kovetkezo formatumban: \l{Az ido 17
    ora 23 perc.}. Eredmenyt nem ad es bemenete sincsen.
    \item \l{beallit} Bevesz parameterkent ket egesz szamot es beallitja oket a jelenlegi idonek. Eredmenyt
    nem ad.
    \item \l{ora} Ez a metodus visszaadja az ora valtozo jelenlegi erteket, parametere nincsen.
\end{itemize}

\newpage

\section{Osztalyokrol Bovebben}

\paragraph{Bevezetes}

Az elozo fejezetben az osztalyokkal kapcsolatos alapveto szabalyokat es koncepciokat fektettuk le, osztalyvaltozok es
metodusok deklaralasat, hasznalatat. Ebben a fejezetben ezeket a koncepciokat bovitjuk ki uj, opcionalis
lehetosegekkel amivel jobb, komplexebb adatmodelleket hozhatunk letre.

\subsection{Konstruktor es a this kulcsszo}

Eddig amikor egy peldanyt hoztunk letre egy valtozobol, a kovetkezo keppen tettuk:

\begin{lstlisting}[language=Java, caption=Peldanyositas es osztalyvaltozok beallitasa]
Kutya kutya1 = new Kutya();
kutya1.nev = "Buksi";
kutya1.kor = 5;
kutya1.oltott = true;
\end{lstlisting}

Szepen olvashato es ertheto a kod, viszont lathatoan sokat kell irnunk ahhoz, hogy ezt a 3 rovid adatot elmentsuk a
peldanyunkban, ezert jo lenne, ha rovidebben ezt le tudnank irni. Az is jo lenne, hogy ha megszabok egy jellemzot egy
osztalyomban, tudjam szabalyozni, hogy azt a jellemzot kotelezoen minden peldanynak be kelljen allitani. Ezzel
elkerulhetnem, hogy veletlenul valamikor kor vagy nev nelkul hozzak letre egy kutya peldanyt. Ezekre es szamos mas
problemara is a konstruktor a megoldas. A konstruktor az egy olyan metodus ami a peldanyositaskor hivodik meg. A
kovetkezo peldaban meghatarozok egy konstruktort a Kutya osztalyomban, majd az uj konstruktorom segitsegevel ujra
letrehozom ugyanazt a peldanyt amit az elozo kodreszletben.

\begin{lstlisting}[language=Java, caption=Konstruktor pelda]
class Kutya {
    String nev;
    int kor;
    boolean oltott;

    Kutya (String nev, int kor, boolean oltott) {
        this.nev = nev;
        this.kor = kor;
        this.oltott = oltott;
    }
}
...
Kutya kutya1 = new Kutya("Buksi", 5, true);
\end{lstlisting}

A fenti peldaban figyeld meg a konstruktor felepiteset. Mint lathatod, nincs visszateresi erteke, csak neve, ami
pedig pontosan megegyezik az osztaly nevevel. Ugyanugy vannak parameterei mint barmilyen masik metodusnak, ebben az
esetben minden osztalyvaltozohoz egy darab. Mivel az osztalyvaltozok es a konstruktor metodus parameterei ugyanazokat
az elnevezeseket kaptak, igy a \l{this} kulcsszoval tudom megkulomboztetni a kettot. Tehat a \l{this
.valtozo} vagy \l{this.metodus} mindig a peldany sajat osztalyvaltozojara vagy metodusara mutat. Ez a
kulcsszo a legtobb esetben opcionalis, de a fenti konstruktorban kotelezo, mert kulonben nem tudna Java megmondani,
hogy az ertekadas baloldalan az osztalyvaltozora gondoltunk, es nem a parameterkent kapott valtozora.

A kovetkezo peldaban olyan konstruktort hozunk letre a \l{Jarmu} osztalyhoz, ami a \l{tulajdonosNeve}
valtozot egy parameter helyett mindig arra allitja be, hogy \l{"Nincs Tulajdonos"}, illetve minden
peldanyositaskor a konzolon ertesiti a felhasznalot, hogy egy uj jarmu jott letre.

\begin{lstlisting}[language=Java, caption=Konstruktor pelda]
class Jarmu {
    String rendszam;
    String gyarto;
    String modell;
    String tulajdonosNeve;

    Jarmu (String rendszam, String gyarto, String modell) {
        this.rendszam = rendszam;
        this.gyarto = gyarto;
        this.modell = modell;
        this.tulajdonosNeve = "Nincs Tulajdonos";
        System.out.println("Egy uj autot hoztunk letre!");
    }
}
\end{lstlisting}

\newpage

\subsection{Feladatok}

\subsubsection{Feladat}

Adj hozza egy teljes konstruktort a kovetkezo osztalyhoz

\begin{lstlisting}[language=Java, caption=Feladat]
class Tanulo {
    int azonosito;
    String nev;
    String osztaly;

    ...
}
\end{lstlisting}

\subsubsection{Feladat}

Adj hozza egy olyan konstruktort a kovetkezo osztalyhoz, ami 4db \l{String} parametert ker be es azok alapjan
allitja be
a \l{szamok} valtozot.

\begin{lstlisting}[language=Java, caption=Feladat]
class Album {
    String[] szamok;

    ...
}
\end{lstlisting}

\newpage

\section{Objektum tombok}

Egy objektum tobb kulonbozo erteket tarol amik egy bizonyos valamit jellemeznek, tehat ugy viselkeldik mint egy
tablazatban egy sor. Ha pedig sok objektumot veszunk (ugyanabbol az osztalybol) akkor pedig egy tablazathoz
hasonlithato strukturat kapunk. Sajat osztalyokkal valo tombok keszitesere szinten ugyanazok a szabalyok vonatkoznak,
mint barmilyen mas adattipusra.

Ebben a peldaban egy 10 elembol allo \l{Kutya} tombot hozunk letre. Ha a tombot a lenti modon inicializaljuk,
\l{null} ertekekkel lesz feltoltve.

\begin{lstlisting}[language=Java, caption=Kutya tomb keszitese - 1. modszer]
Kutya[] kutyak = new Kutya[10];
\end{lstlisting}

Ha szeretnenk rogton ertekekkel feltolteni egy objektum tombot, megtehetjuk, hogy felsoroljuk az ertekeket.

\begin{lstlisting}[language=Java, caption=Kutya tomb keszitese - 2. modszer]
Kutya kutya1 = new Kutya("Buksi", 5, true);
Kutya kutya2 = new Kutya("Folti", 2, true);
Kutya kutya3 = new Kutya("Tapi", 9, false);

Kutya[] kutyak = {kutya1, kutya2, kutya3};
\end{lstlisting}

Ha az ertekek meg nem leteznek, inicializalhatjuk oket a felsorolason belul is.

\begin{lstlisting}[language=Java, caption=Kutya tomb keszitese - 3. modszer]
Kutya[] kutyak = {new Kutya("Buksi", 5, true),
                new Kutya("Folti", 2, true),
                new Kutya("Tapi", 9, false)};
\end{lstlisting}

\newpage

\subsection{Objektum Tomb Feldolgozasa}

Miutan letrehoztunk egy objektumokat tartalmazo tombot, azt az eddig tanult modszerekkel kezelhetjuk, peldaul \l{for}
ciklusokkal bejarhatjuk es modosithatjuk oket vagy statisztikakat szamolhatunk beloluk. Tetszes szerint valaszthatunk
\l{for} es \l{for-each} ciklusok kozott, de a legtobb esetben nem vagyunk kivancsiak az elemek indexeire es nem
terunk el az alap bejarasi modszertol, igy a \l{for-each} jobb  valasztas lesz.

\begin{lstlisting}[language=Java, caption=A kutya tomb amivel dolgozni fogunk a kovetkezo peldakban]
Kutya[] kutyak = {new Kutya("Buksi", 5, true),
                new Kutya("Folti", 2, true),
                new Kutya("Tapi", 9, false)};
\end{lstlisting}

\subsubsection{Bejaras, Megjelenites}

\begin{lstlisting}[language=Java, caption=Kutyak adatainak megjelenitese]
System.out.println("nev\tkor\toltott");
for(Kutya k : kutyak) {
    System.out.println(k.nev + "\t" + k.kor + "\t" + k.oltott);
}
\end{lstlisting}

\subsubsection{Extrem Ertek Kereses}

\begin{lstlisting}[language=Java, caption=Legidosebb kutya megkeresese]
Kutya legidosebb = kutyak[0];

for(Kutya k : kutyak) {
    if(k.kor > legidosebb) {
        legidosebb = k.kor;
    }
}

System.out.println("A legidosebb kutya " + legidosebb.nev
            + " aki " + legidosebb.kor + " eves." );
\end{lstlisting}

\newpage

\subsection{Feladatok}

A lenti feladatokban a kovetkezo osztallyal fogunk dolgozni:

\begin{lstlisting}[language=Java, caption=Feladat - Termek osztaly]
class Termek {
    int azonositoSzam;
    String megnevezes;
    int ar;
    String kategoria;

    Termek(int azonositoSzam, String megnevezes,
                int ar, String kategoria) {
        this.azonositoSzam = azonositoSzam;
        this.megnevezes = megnevezes;
        this.ar = ar;
        this.kategoria = kategoria;
    }

}

\end{lstlisting}

\subsubsection{Feladat}
A \l{Termek} osztalybol keszits peldanyokat a kovetkezo adatokkal es helyezd el oket egy \l{termekek} nevu tombben.

\begin{table}[H]
    \begin{tabular}{|l|l|l|l|}
        \hline
        \textbf{Azonosito} & \textbf{Megnevezes} & \textbf{Ar} & \textbf{Kategoriak}    \\ \hline
        10001              & Alma                & 20          & elelmiszer             \\ \hline
        10002              & Kenyer              & 32          & elelmiszer             \\ \hline
        10003              & C-Vitamin           & 19          & egeszseg               \\ \hline
        10004              & Billentyuzet        & 262         & elektronika            \\ \hline
        10005              & Telefon Tolto       & 220         & elektronika            \\ \hline
    \end{tabular}
    \caption{Termekek}
    \label{tab:termekek}
\end{table}

\newpage

\subsubsection{Feladat}

Egy ciklus segitsegevel ird ki az osszes termek adatait a kovetkezo formaban:

\begin{lstlisting}[language=Java, caption=Feladat]
Azonosito   Megnevezes          Ar(Ft)      Kategoriak
10001       Alma                20          elelmiszer
10002       Kenyer              32          elelmiszer
10003       C-Vitamin           19          egeszseg
10004       Billentyuzet        262         elektronika
10005       Telefon Tolto       220         elektronika
\end{lstlisting}

Egy ciklus segitsegevel ird ki az osszes elelmiszer kategoriaban levo termek adatait a kovetkezo formaban:

\begin{lstlisting}[language=Java, caption=Feladat]
Elelmiszer kategoria termekei:
Azonosito   Megnevezes          Ar(Ft)
10001       Alma                20
10002       Kenyer              32
\end{lstlisting}

\subsubsection{Feladat}

Egy ciklus segitsegevel szamold ki a termekek atlagarat es ird ki a konzolra a kovetkezo formaban:

\begin{lstlisting}[language=Java, caption=Feladat]
A termekek atlagara 110.6 Ft.
\end{lstlisting}

\subsubsection{Feladat}

Egy ciklus segitsegevel keresd meg es jelenitsd meg a legdragabb termek adatait

\begin{lstlisting}[language=Java, caption=Feladat]
A legdragabb termek:
Azonosito   Megnevezes          Ar      Kategoriak
10004       Billentyuzet        262     elektronika
\end{lstlisting}

\end{document}