\documentclass{article}

\usepackage[utf8]{inputenc}
\usepackage[hungarian]{babel}
\usepackage{hyperref}
\usepackage{xcolor}
\usepackage{listings}
\renewcommand{\lstlistingname}{Kodreszlet}%
\lstset{language=Java,
backgroundcolor=\color[HTML]{ebebeb},
keywordstyle={\bfseries \color[HTML]{5d00ff}},
frame=single,
basicstyle=\footnotesize\ttfamily,
captionpos=b,
tabsize=2,
numbers=left,
aboveskip=2em,
belowskip=1em
}
\usepackage{graphicx}
\usepackage{float}
\restylefloat{table}
\let\l\lstinline

\title{%
Java SE Vizsga\\
\large Adatbazis- es Halozatkezeles\\
\large Rendelkezesre allo ido: 90 perc}

\author{Szabo Daniel\\daniel.szabo99@outlook.com}

\date{\today}

\begin{document}

    \maketitle

    \section{Utasitasok}

    \subsection{Projekt megnyitasa}

    A vizsgahoz a kovetkezo fajlokra van szukseged:
    \begin{itemize}
        \item \l{pingpong.Ping.java},
        \item \l{pingpong.Pong.java},
        \item \l{termekek.Adatbazis.java},
        \item \l{termekek.Termek.java},
        \item \l{termekek.Main.java},
        \item \l{termekek.termekek.sql},
        \item legfrissebb JDBC meghajto.
    \end{itemize}
    Ezeket egy uj Eclipse Java projekt-ben helyezzuk el. A projekt neve legyen \l{JavaSEAdatbazisEsHalozatokVizsga_VezeteknevKeresztnev}. A projekt-ben hozzunk letre a \l{pingpong} es \l{termekek} csomagot es azokban helyezzuk el az osztalyainkat. Ne felejtsd el a csomag megnevezeset az osztalyokban is modositani.

    \subsection{Feladatok megoldasa}

    \begin{itemize}
        \item A megoldas soran kovessuk az objektum orientalt programozas iranyelveit, torekedjuk osztalyokat es metodusokat hasznalni a megoldashoz.
        \item Nem mukodo kodot illetve felesleges kommenteket ne hagyjunk a projektben
    \end{itemize}

    \subsection{Megoldas beadasa}

    A teljes projektet egy zip allomanykent mentsuk el. Ezt megtehetjuk a File $\rightarrow$ Export $\rightarrow$ Archive File menuponttal. A tomoritett allomany neve legyen a projekt nevevel azonos. Beadas elott csomagold ki a projektet es ellenorizd, hogy mukodik a megoldasod, majd add be a zip allomanyt.

\section{Feladatok}

Mielott hozzakezdenel a feladatok megoldasahoz, figyelmesen olvasd el a kodot amit kaptal es ertelmezd az osztalyok elemeit, keresd meg a hianyzo reszeket es a feladatok helyeit.

\subsection{Feladat (4 pont)}
Egeszitsd ki a \lstinline{Ping} osztaly main metodusaban levo Socket kapcsolatot ugy, hogy a kapcsolat kiepitese utan rogton kuldjon egy String "ping" uzenetet a szervernek es a konzolra is irja ki, hogy "ping", majd varjon bejovo uzenetekre. Ha a bejovo uzenet "pong", irja ki a konzolra, hogy "pong", ha nem ez a valaszuzenet, akkor irja ki a konzolra, hogy "Helytelen Pong valaszuzenet fogadva".

\subsection{Feladat (4 pont)}
Egeszitsd ki a \lstinline{Pong} osztaly main metodusaban levo Socket kapcsolatot ugy, hogy a kapcsolat kiepitese utan rogton irja ki a konzolra, hogy "Bejovo kapcsolat fogadva:" majd utana a bejovo kapcsolat IP cimet. Ez utan rogton olvassa be az elso bejovo uzenetet. Ha az uzenet "ping", valaszoljon "pong"-gal es irja ki a konzolra, hogy "Pong visszakuldve". Ha az uzenet nem "ping", irja ki a konzolra, hogy "Helytelen ping erkezett".

Az elozo ket feladat megoldasa utan a Pong szerver es a Ping kliens inditasa utan a kovetkezo kimenetet kell lassuk:

\begin{lstlisting}[language=Java, caption=Ping]
ping
pong
\end{lstlisting}

\begin{lstlisting}[language=Java, caption=Pong]
Pong varakozik...
Bejovo kapcsolat fogadva: /127.0.0.1
Pong visszakuldve.
Pong varakozik...
\end{lstlisting}


\subsection{Feladat (2 pont)}
Az \lstinline{Adatbazis} osztaly \lstinline{csatlakozas()} metodusat egeszitsd ki ugy, hogy a megfelelo helyen toltse be a JDBC meghajtot.

\subsection{Feladat (2 pont)}

A termekek.sql segitsegevel hozd letre a \l{termekek } adatbazist es a \l{termek} tablat, majd hozz letre egy felhasznalot aki hozzafer ehhez a tablahoz es lekerdezesi joga van.
Az \lstinline{Adatbazis} osztaly \lstinline{csatlakozas()} metodusat egeszitsd ki ugy, hogy a megfelelo helyen csatlakozzon az adatbazisodhoz. A kiepitett kapcsolatot a globalis \l{conn} valtozo kapja meg.

\subsection{Feladat (3 pont)}
Az \lstinline{Adatbazis} osztaly \lstinline{lekerdezes()} metodusat egeszitsd ki ugy, hogy a \l{termekek} listaba a kapott \l{ResultSet} objektumbol betolti a lekerdezes eredmenyekent beerkezett termekeket.

\subsection{Feladat (1 pont)}
Az \lstinline{Adatbazis} osztaly \lstinline{bezaras()} metodusat egeszitsd ki ugy, hogy zarja le az adatbaziskapcsolatot. Ha sikertelen volt a muvelet, jelenitsen meg egy megfelelo hibauzenetet.

\subsection{Feladat (2 pont)}
Az \lstinline{Main} osztaly \lstinline{main()} metodusat egeszitsd ki ugy, hogy kerdezze le az \l{Adatbazis} osztalyunk segitsegevel az osszes termek adatait majd azokat jelenitse meg.

\newpage

\subsection{Feladat (2 pont)}
Az \lstinline{Main} osztaly \lstinline{main()} metodusat egeszitsd ki ugy, hogy kerdezze le az \l{Adatbazis} osztalyunk segitsegevel a 100 forintnal olcsobb termekek adatait ar alapjan novekvo sorrendben majd azokat jelenitse meg.

Az elozo ket feladat megoldasa utan a \l{main()} metodus futtatasa utan a kovetkezo kimenetet kell lassuk:

\begin{lstlisting}[language=Java, caption=Main.main() kimenet]
Csatlakozas...
JDBC meghajto sikeresen betoltve.
Sikeresen csatlakozva a MySQL adatbazishoz.
Minden termek:
Termek{id=1, megnevezes='piros alma', ar=50}
Termek{id=2, megnevezes='zold alma', ar=55}
Termek{id=3, megnevezes='banan', ar=40}
...
Termek{id=35, megnevezes='macskaalom', ar=400}
Termek{id=36, megnevezes='kullancsriaszto', ar=240}
Termek{id=37, megnevezes='parna', ar=500}
Olcso termekek:
Termek{id=7, megnevezes='zsemle', ar=34}
Termek{id=3, megnevezes='banan', ar=40}
Termek{id=1, megnevezes='piros alma', ar=50}
Termek{id=6, megnevezes='kifli', ar=50}
Termek{id=2, megnevezes='zold alma', ar=55}
MySQL kapcoslat sikeresen lezarva.
\end{lstlisting}



\end{document}