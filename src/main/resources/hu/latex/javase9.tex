\documentclass{article}

\usepackage[utf8]{inputenc}
\usepackage[hungarian]{babel}
\usepackage{hyperref}
\usepackage{xcolor}
\usepackage{listings}
\renewcommand{\lstlistingname}{Kodreszlet}%
\lstset{language=Java,
backgroundcolor=\color[HTML]{ebebeb},
keywordstyle={\bfseries \color[HTML]{5d00ff}},
frame=single,
basicstyle=\footnotesize\ttfamily,
captionpos=b,
tabsize=2,
numbers=left,
aboveskip=2em,
belowskip=1em
}
\usepackage{graphicx}
\usepackage{float}
\restylefloat{table}
\let\l\lstinline

\title{%
Java SE 9 \\
\large Generikusok}

\author{Szabo Daniel\\daniel.szabo99@outlook.com}

\date{\today}

\begin{document}

\maketitle
\begin{abstract}
Ebben a feladatsorban megismerkedunk a generikusokkal, amik lehetove teszik szamunkra a kodunk altalanositasat es felkeszit egyedi adatszerkezetek keszitesere.
\end{abstract}

\newpage

\tableofcontents{}

\newpage

\section{Generikusok elonyei}
\paragraph{Bevezetes}

Adott az alabbi \l{Doboz} osztaly, ami egyetlen erteket tarol, ennek a valtozonak a neve legyen csak \l{ertek}. Java-ban az osztalyvaltozoknak kotelezo tipust meghataroznunk, igy ha a dobozunkat univerzalissa szeretnenk tenni, a tipusa \l{Object} kell legyen, amibe utana az oroklodes szabalyai szerint barmilyen objektumot eltarolhatunk.

\begin{lstlisting}[language=Java, caption=Doboz osztaly]
public class Doboz {
	Object ertek;

	public Object getErtek() {
		return ertek;
	}

	public void setErtek(Object ertek) {
		this.ertek = ertek;
	}
}
\end{lstlisting}

Amikor ebben ezt az osztalyt peldanyositunk, barmilyen objektumot tarolhatunk benne, de az ertek hasznalatahoz es lekerdezeshez szukito tipuskonverziora van szuksegunk. A lenti peldaban String tipussal hasznaljuk a dobozt, es ez a kod hibatlanul mukodik, azonban rogton felismerjuk, hogy ez csak addig lesz helyes, amig emlekszunk arra, hogy ez a doboz String-et tarol. Ha sok dobozunk van, nekunk kell emlekeznunk mindig arra, hogy melyik doboz mit tarol, kulonben helytelen tipuskonverziot vegezhetunk es hibat kaphatunk, illetve ugyanez tortenhet, ha veletlenul nem String erteket tarolunk el a dobozban.

\begin{lstlisting}[language=Java, caption=Doboz hasznalata String-gel]
Doboz doboz = new Doboz();
doboz.setErtek("Hello");
String ertek = (String) doboz.getErtek();
\end{lstlisting}

Ebben a helyzetben lenne hasznos az, ha a \l{Doboz} osztaly minden peldanyanak kulon deklaralhatnank, milyen tipusa legyen, amit generikusokkal megtehetunk.

\newpage

\section{Generikusok hasznalata}

A kovetkezo peldaban a \l{Doboz} osztalyt ugy modositottuk, hogy \l{Object} helyett egy \l{T} tipusparameter altal megszabott tipusu erteket tarol, illetve ez a tipus jelenik meg a \l{getErtek()} visszateresi tipusakent es a \l{setErtek()} parameterenek tipusakent is.

\begin{lstlisting}[language=Java, caption=Generikus Doboz osztaly]
public class Doboz<T> {
	T ertek;

	public T getErtek() {
		return ertek;
	}

	public void setErtek(T ertek) {
		this.ertek = ertek;
	}
}
\end{lstlisting}

Innentol kezdve amikor a \l{Doboz} osztalybol peldanyositunk, az osztaly neve mogotti \l{<TipusNeve>} segitsegevel adhatjuk meg egy osztaly (primitiv tipus nem hasznalhato) nevet, ami erre a valtozora vegleges lesz. Mivel a \l{T} parameter a metodusokra is alkalmazva lett, a \l{getErtek()} az alabbi peldaban String ertekkel ter vissza, tehat tipuskonverziora nincs szuksegunk.

\begin{lstlisting}[language=Java, caption=Generikus Doboz hasznalata String-gel]
Doboz<String> doboz = new Doboz<>();
doboz.setErtek("Hello");
String ertek = doboz.getErtek();
\end{lstlisting}

\newpage

\subsection{Generikusok Szukitese}

Amikor egy generikus tipust letrehozunk, alapvetoen barmilyen tipust megadhatunk a peldanyoknak parameterkent, illetve amikor megsem adunk meg neki tipusparametert, akkor a legtagabb szuloosztaly lesz a tipusa az peldanynak, a \l{Doboz} eseteben tehat \l{Object}. Sok esetben azonban nem teljesen mindegy, milyen tipusokat engedelyezunk az osztalyunkban, igy megjelolhetunk egy osztalyt vagy interfeszt amit es aminek altipusait elfogadunk.

Az alabbi peldaban a \l{SzamDoboz} egy olyan osztaly aminek a tipusparametere a \l{Number} osztaly alosztalya kell legyen, tehat \l{SzamDoboz<String>} peldaul forditasi hibat okoz. Ez azt is jelenti, hogy az alapveto legtagabb tipusa ennek a parameternek \l{Number} lesz, es amikor tipusparameter nelkul peldanyositjuk a \l{SzamDoboz} osztalyt, akkor is csak szamokkal mukodik, de akkor minden fele szammal.

\begin{lstlisting}[language=Java, caption=SzamDoboz osztaly]
public class SzamDoboz<T extends Number> {
	T ertek;

	public T getErtek() {
		return ertek;
	}

	public void setErtek(T ertek) {
		this.ertek = ertek;
	}
}
\end{lstlisting}

\subsection{Generikusok Metodusok}

Vannak olyan esetek, amikor statikus metodusnak lenne szuksegunk egy generikus valtozatara. Ilyenkor, mivel a metodus statikus, a peldanyositas folyamata, amikor peldanyoknaknak a tipusparameteret meghatarozzuk, nem hasznos.

A lenti peldaban a statikus \l{visszadob} metodus is generikus, azonban a tipusa nem az osztalyatol fugg, hanem a kapott parameter tipusatol, tehat ha String bemenetet kap, String objektummal fog visszaterni, ami azt jelenti, hogy univerzalissa tehetem a metodusom szukito tipuskonverzio nelkul.

\begin{lstlisting}[language=Java, caption=Generikus metodus]
static <T> T visszadob(T ertek) {
	return ertek;
}
\end{lstlisting}

Generikus metodusok eseteben is van lehetosegunk szukiteni a hasznalhato tipusok listajat, az alabbi peldaban peldaul csak \l{List} implementaciokat lehet hasznalni. Mivel T mindig egy \l{List} lesz, igy meg tudjuk hivni az \l{isEmpty()} metodust, es ha igazat ad eredmenyul, \l{null} lesz az eredmeny, minden mas esetben bejovo erteket adjuk vissza.

\begin{lstlisting}[language=Java, caption=Generikus metodus]
static <T extends List> T visszadob(T ertek) {
	if(ertek.isEmpty()) {
		return null;
	} else {
		return ertek;
	}
}
\end{lstlisting}


\subsection{Tobb Tipusparameter Hasznalata}

Van lehetosegunk egyszerre tobb tipusparametert is megadni egy osztalynak. Az alabbi \l{SzamozottErtek} osztaly \l{SZ} tipusparametere egy szam kell legyen, \l{E} viszont barmi lehet.

\begin{lstlisting}[language=Java, caption=Osztaly tobb tipusparameterrel]
public class SzamozottErtek<SZ extends Number, E> {
	SZ szam;
	E ertek;

	public SzamozottErtek(SZ szam, E ertek) {
		this.szam = szam;
		this.ertek = ertek;
	}

	public SZ getSzam() {
		return szam;
	}

	public E getErtek() {
		return ertek;
	}
}
\end{lstlisting}
A fenti \l{SzamozottErtek} osztalyt a kovetkezo keppen tudjuk hasznalni:

\begin{lstlisting}[language=Java, caption=SzamozottErtek osztaly hasznalata]
SzamozottErtek<Integer, String> ertek1 =
	new SzamozottErtek<>(1, "Juliska");
SzamozottErtek<Integer, String> ertek2 =
	new SzamozottErtek<>(2, "Jancsika");
\end{lstlisting}

\newpage

\section{Feladatok}

\subsection{Feladat}

Mi tortenik, ha
\begin{itemize}
    \item egy generikus tipussal rendelkezo osztaly nem kap tipusparametert,
    \item egy generikus tipussal nem rendelkezo osztaly kap tipusparametert,
	\item egy generikus osztaly tipusparameteret az osztalyban nem hasznaljuk,
	\item egy generikus osztaly tipusparameteret probaljuk peldanyositani,
	\item egy generikus osztaly tipusparameterevel probalunk tipuskonverziot vegezni,
	\item egy generikus osztaly tipusparameteren probalunk metodust vagy valtozot elerni,
	\item egy generikus osztalyt kibovitunk,
	\item egy interfesznek adunk generikus tipusparametert,
	\item egy generikus interfeszt implementalunk,
	\item egy generikus osztalyban egy generikus, statikus metodust hozunk letre?
\end{itemize}

\subsection{Feladat}

Figyelmesen olvasd el es elemezd a \l{java.util.List} interfesz implementaciojat, figyeld meg es ertelmezd, hogy hogyan hasznalja a generikusokat.

\subsection{Feladat}

Figyelmesen olvasd el es elemezd a \l{java.util.ArrayList} osztaly implementaciojat, figyeld meg es ertelmezd, hogy hogyan hasznalja a generikusokat.

\subsection{Feladat}

Hozz letre egy listat, amiben a \l{SzamDoboz} osztalyunk \l{Double} formajanak elemeit tarolhatjuk, es tarolj el benne harom \l{SzamDoboz} peldanyt.

\subsection{Feladat}

Implementald a \l{SzamDoboz} osztalyban a \l{Comparable} interfeszt ugy, hogy minden \l{SzamDoboz} mas \l{SzamDoboz} peldanyokkal osszehasonlithato legyen az eltarolt ertekeik alapjan.

\end{document}

