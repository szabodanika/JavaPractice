\documentclass{article}

\usepackage[utf8]{inputenc}
\usepackage[hungarian]{babel}
\usepackage{hyperref}
\usepackage{xcolor}
\usepackage{listings}
\renewcommand{\lstlistingname}{Kodreszlet}%
\lstset{language=Java,
backgroundcolor=\color[HTML]{ebebeb},
keywordstyle={\bfseries \color[HTML]{5d00ff}},
frame=single,
basicstyle=\footnotesize\ttfamily,
captionpos=b,
tabsize=2,
numbers=left,
aboveskip=2em,
belowskip=1em
}
\usepackage{graphicx}
\usepackage{float}
\restylefloat{table}
\let\l\lstinline

\title{%
Java SE 7 \\
\large File IO}

\author{Szabo Daniel\\daniel.szabo99@outlook.com}

\date{\today}

\begin{document}

\maketitle
\begin{abstract}
Ebben a feladatsorban megtanulunk a fajlrendszerbol adatot betolteni illetve adatot elmenteni es ezeknek a muveleteknek egy praktikus felhasznalasat.
\end{abstract}

\newpage

\tableofcontents{}

\newpage

\section{Alap fajlmuveletek}
\paragraph{Bevezetes}

Java-ban rengeteg eszkoz all rendelkezesunkra fajlok kezelesehez es szamos modon hasznalhatjuk ezeket, de ebben a feladatsorban ket eszkozre fogunk koncentralni ket kulonbozo esetre, illetve mindket eszkozt hasonlo eljarassal fogjuk alkalmazni. Eloszor a \l{FileInputStream} osztalyt fogjuk hasznalni, aminek segitsegevel egy fajlt byte-onkent kezelhetunk.

\subsection{Falj olvasas (byte-okkent)}

A kovetkezo peldaban peldanyositunk egy \l{FileInputStream} objektumot eloszor, ami \l{FileNotFoundException}-t dob. Ezt kotelezoen lekezeljuk egy \l{try-catch} blokk segitsegevel. A \l{FileInputStream} konstruktorakent a megnyitando fajl nevet (pontosabban eleresi utjat a projekt gyokerkonyvtarahoz vagy a futo program helyehez kepest relativan) adjuk meg egy \l{String} valtozokent. Ezutan a \l{readAllBytes()} metodus segitsegevel egy \l{byte} tomb formajaban beolvassuk a fajlt teljes egeszeben. Ez a metodus \l{IOException}-t dobhat, ezert ezt is le kell kezelnunk a tanult modszerek egyikevel. Ebben az esetben kulon \l{catch} blokkot kapott. Ezutan a \l{String} osztaly segitsegevel a beolvasott \l{byte} tombot szovegge alakitjuk es kiirjuk a konzolra, hogy ha biztosak vagyunk benne, hogy a beolvasott adat szoveges. Vegul a \l{close()} metodussal bezarjuk a fajlt, ami egy nagyon fontos lepes, amire azert van szukseg, hogy miutan mar nem dolgozunk az allomannyal, az operacios rendszer tisztaban legyen azzal, hogy gond nelkul engedheti mas folyamatoknak, hogy hasznaljak a fajlt, mert mi mar kesz vagyunk a fajlmuveletunkkel.

\begin{lstlisting}[language=Java, caption=FileInputStream hasznalata]
try {
    FileInputStream fis = new FileInputStream("adat.txt");
    byte[] beolvasottAdat = fis.readAllBytes();
    System.out.println(new String(beolvasottAdat));
    fileInputStream.close();
} catch (FileNotFoundException fileNotFoundException) {
    fileNotFoundException.printStackTrace();
} catch (IOException ioException) {
    ioException.printStackTrace();
}
\end{lstlisting}

\newpage

Egy elegansabb formaja a \l{FileInputStream}-hez hasonlo objektumok hasznalatanak a \l{try-with-resources} blokk, ami automatikusan lezarja a fajlt, amint a \l{try-catch} blokkunk lezarult.

\begin{lstlisting}[language=Java, caption=Try-with-resources blokk hasznalata]
try (FileInputStream fis = new FileInputStream("adat.txt")){
    byte[] beolvasottAdat = fis.readAllBytes();
} catch (FileNotFoundException fileNotFoundException) {
    fileNotFoundException.printStackTrace();
} catch (IOException ioException) {
    ioException.printStackTrace();
}
\end{lstlisting}

\subsection{Falj iras (byte-okkent)}

A \l{FileInputStream}-hez hasonloan mukodik a \l{FileOutputStream}, ami egy parameterkent kapott eleresi uton levo fajlt nyit meg, es a \l{write()} metodussal \l{byte} tombot tud beleirni. Ha a fajl nem letezik, akkor letre fogja hozni, tehat \l{FileNotFoundException}-t nem dob, csak abban az esetben, ha a megjelolt mappa sem letezik, vagy valamilyen okbol (peldaul engedely hianya) nem tudta letrehozni a fajlt. A kovetkezo peldaban szinten a \l{try-with-resources} blokkot hasznaljuk, es egy \l{String} valtozo \l{getBytes()} metodusanak segitsegevel irunk szoveget az adat.txt fajlba.

\begin{lstlisting}[language=Java, caption=Fajlba iras FileOutputStream segitsegevel]
try (FileOutputStream fos = new FileOutputStream("adat.txt")){
    fos.write("Hello".getBytes());
} catch (IOException ioException) {
    ioException.printStackTrace();
}
\end{lstlisting}

\newpage

\subsection{Falj olvasas (String-kent)}

A \l{FileInputStream} kitunoen alkalmazhato szoveges beolvasasra is, de soronkent beolvasni vagy specialis kodolasu szoveggel dolgozni viszonylag trukkos lehet, igy amikor kifejezetten szoveget szeretnenk beolvasni, \l{BufferedReader} osztalyt hasznalunk. Mivel a \l{BufferedReader} nem kifejezetten csak fajlmuveletekhez, hanem akar halozati muveletekre is alkalmas, igy egy \l{FileReader} objektumot kap parameterkent, es majd csak az fogja tartalmazni a hivatkozas a fajlunkra. A hivatkozas a fajlra tortenhet egy \l{File} objektummal vagy egyszeruen String-kent is.

\begin{lstlisting}[language=Java, caption=BufferedReader hasznalata]
try (BufferedReader br = new BufferedReader(
        new FileReader(new File("adat.txt")))){
    System.out.println(br.readLine());
} catch (IOException ioException) {
    ioException.printStackTrace();
}
\end{lstlisting}

Vagy ugyanez a \l{File} hivatkozas nelkul:

\begin{lstlisting}[language=Java, caption=BufferedReader hasznalata]
try (BufferedReader br = new BufferedReader(
        new FileReader("adat.txt"))){
...
\end{lstlisting}

A fenti peldaban szinten az adat.txt fajlt olvassuk be, viszont a \l{BufferedReader} osztaly \l{readLine()} metodusat hasznalva, ami \l{byte} tomb helyett \l{String} objektummal ter vissza. Fontos megjegyezni azonban, hogy ez a metodus csak az elso sortoresig olvassa be a fajlt, ha tovabb szeretnem olvasni, akkor ujbol meg kell hivnom, es minden hivaskor a kovetkezo sorig olvas, egeszen addig, amig mar nincs mit beolvasni, akkor pedig \l{null}-al ter vissza. A kovetkezo pelda egy \l{while} ciklus segitsegevel addig olvassa es irja ki konzolra a sorokat, amig el nem erjuk a fajl veget.

\begin{lstlisting}[language=Java, caption=BufferedReader hasznalata]
try (BufferedReader br = new BufferedReader(
        new FileReader(new File("adat.txt")))){
    String sor = null;
    while ((sor = br.readLine()) != null) {
        System.out.println(sor);
    }
} catch (IOException ioException) {
    ioException.printStackTrace();
}
\end{lstlisting}

\newpage

\subsection{Falj iras (String-kent)}

A \l{BufferedWriter} szinten hasonlit a \l{BufferedReader}-re, ugyanugy peldanyositjuk es ugyanugy \l{String}-ekkel dolgozunk. A lenti pelda beleirja az adat.txt-be, hogy "Hello".

\begin{lstlisting}[language=Java, caption=Fajlba iras BufferedWriter segitsegevel]
try (BufferedWriter br = new BufferedWriter(
        new FileWriter(new File("adat.txt")))){
    br.write("Hello");
} catch (IOException ioException) {
    ioException.printStackTrace();
}
\end{lstlisting}

Fontos megjegyeznunk azt, hogy amikor fajlba irunk, alapvetoen az eredeti tartalom felul lesz irva. Ha nem felulirni szeretnenk az adatot, hanem a vegere irni, akkor azt is megtehetjuk, ha egy \l{true} erteket is megadunk a \n{FileWriter} konstruktoraban:

\begin{lstlisting}[language=Java, caption=Fajl vegere iras BufferedWriter segitsegevel]
try (BufferedWriter br = new BufferedWriter(
new FileWriter(new File("adat.txt"), true))){
\end{lstlisting}

A kovetkezo pelda tobb sort ir a fajlba, 0-tol 9-ig a szamokat, mindet uj sorba. Sortorest a \l{newLine()} metodussal irhatunk, vagy String-ek eseteben a szokasos \l{"\n"}-el. Figyeljunk oda arra, hogy a \l{write()} metodusnak ne adjunk parameterkent szamot, amikor a szamot szovegkent szeretnenk a fajlba irni, kulonben a szamot \l{byte}-kent fogja irni, ami visszakonvertalva String-ge valoszinuleg nem a szam karaktere lesz, hanem egy specialis karakter vagy egy betu, a karakterkodolastol fuggoen, ami alapvetoen az operacios rendszer altal van meghatarozva es a legtobb esetben UTF-8 lesz. Javasolt megismerkedni a gyakori karakterkodolasokkal es kulonbsegeikkel is.

Segitseg: \href{https://hu.wikipedia.org/wiki/Kateg\%C3\%B3ria:Karakterk\%C3\%B3dol\%C3\%A1sok}{Wikipedia - Karakterkodolasok}

\begin{lstlisting}[language=Java, caption=Fajlba iras BufferedWriter segitsegevel]
try (BufferedWriter br = new BufferedWriter(
    new FileWriter(new File("adat.txt")))){
    for (int i = 0; i < 10; i++) {
        br.write(String.valueOf(i));
        br.newLine();
    }
} catch (IOException ioException) {
    ioException.printStackTrace();
}
\end{lstlisting}

\newpage

\section{CSV fajl kezelese}

Ugyan rengeteg potencialis gyakorlati almazasa van a most tanultaknak, mi egy specialis, gyakori esetre fogunk koncentralni: tablazat jellegu, CSV (vesszovel elvalasztott ertekeket tartalmazo) formatumu adat feldolgozasara.

A kovetkezo peldakban vegyuk alapul az adat.csv fajlt aminek a tartalma a kovetkezo:

\begin{lstlisting}[language=Java, caption=Adat.csv]
gyarto,modell,ev
Toyota,Corolla,2008
Toyota,Yaris,2017
Ford,Focus,2011
Ford,Mondeo,2016
\end{lstlisting}

\subsection{CSV fajl beolvasasa}

Az adat.csv fajl a legalapvetobb tablazatformatumot koveti, amelyben a sorok sortoressel vannak elvalasztva, az oszlopok pedig "," karakterrel, ami tetszes szerint lehetne tabulator vagy pontosvesszo is. Az elso sor az oszlopneveket tartalmazza, utana minden sor egy-egy jarmu adatait.

\begin{lstlisting}[language=Java, caption=BufferedReader hasznalata]
try (BufferedReader br = new BufferedReader(
        new FileReader(new File("adat.txt")))){
    String sor = null;
    while ((sor = br.readLine()) != null) {
        System.out.println(sor);
    }
} catch (IOException ioException) {
    ioException.printStackTrace();
}
\end{lstlisting}

A fenti pelda segitsegevel beolvashatjuk az adat.csv fajlt es megjelenithetjuk minden sorat a konzolon, azonban mi szeretnenk kulon kezelni az oszlopokat. A kovetkezo kodreszlet azt mutatja be, hogy a \l{String} osztaly \l{.split()} metodusa segitsegevel minden "," karakternel felbonthatjuk a beolvasott sorunkat, es egy sor ertekeit \l{String} tomb formaban megkaphatjuk, amin utana index alapjan elerhetjuk az adott sorban egy bizonyos oszlop erteket. Mivel a cimsort nem szeretnenk megjeleniteni, a \l{while} ciklus elott egyszer meghivjuk a \l{readLine()} metodust, hogy atugorjuk a fajl elso sorat.

\newpage

\begin{lstlisting}[language=Java, caption=Sorok betoltese tombkent]
try (BufferedReader br = new BufferedReader(
        new FileReader(new File("adat.txt")))){
    br.readLine();
    String sor = null;
    String[] ertekek;
    while ((sor = br.readLine()) != null) {
        ertekek = sor.split(",");
        System.out.println("Gyarto: " + ertekek[0] +
                ", Modell: " + ertekek[1] + ", Ev: " + ertekek[2]);
    }
} catch (IOException ioException) {
    ioException.printStackTrace();
}
\end{lstlisting}

A fenti pelda a kovetkezo kimenetet fogja produkalni:

\begin{lstlisting}[language=Java, caption=Kimenet]
Gyarto: Toyota, Modell: Corolla, Ev: 2008
Gyarto: Toyota, Modell: Yaris, Ev: 2017
Gyarto: Ford, Modell: Focus, Ev: 2011
Gyarto: Ford, Modell: Mondeo, Ev: 2016
\end{lstlisting}

Ahhoz, hogy egy egyszeru megjelenitesnel komplexebb muveleteket vegezhessunk a beolvasott adaton, gyakran celszerubb hagyomanyos objektumokkal dolgozni, igy letrehozunk egy olyan osztalyt, ami a fajl egy soraban levo adatokat tarolni tudja.

\begin{lstlisting}[language=Java, caption=Auto osztaly]
public class Auto {
    private String gyarto, modell;
    private int ev;

    ... teljes konstruktor es get metodusok
}
\end{lstlisting}

A cel tehat hogy a CSV fajl minden sorat egy \l{Auto} objektumba beolvassuk, hogy utana azokon tudjunk muveleteket vegezni. A lenti peldaban letrehozunk egy dinamikus listat ami kepes lesz az \l{Auto} objektumok tarolasara. Azert nem tombot hozunk letre, mert nem tudjuk pontosan hany auto van a fajlunkban. Az elozo peldahoz hasonloan felbontunk minden sort egyedi ertekekre, majd ezeket az ertekeket felhasznalva hozunk letre minden sor utan egy uj \l{Auto} objektumot, aminek minden parameterekent a beolvasott sor egy-egy oszlopaban levo ertekeket adunk meg. Figyelj arra, hogy az ev valtozo \l{int} tipusu, viszont az \l{ertekek} tomb minden erteket \l{String} tipusban tarol, igy at kell alakitanom az \l{Integer.parseInt()} metodussal, mielott a konstruktornak megadhatnam.

Miutan a lenti kod lefutott, egy hagyomanyos listan hagyomanyos objektumaim lesznek, amivel barmilyen korabban tanult muveletet elvegezhetek.

\begin{lstlisting}[language=Java, caption=Sorok betoltese objektumokkent]
List<Auto> autok = new ArrayList<>();
try (BufferedReader br = new BufferedReader(
        new FileReader(new File("adat.txt")))){
    br.readLine();
    String sor = null;
    String[] ertekek;
    while ((sor = br.readLine()) != null) {
        ertekek = sor.split(",");
        autok.add(new Auto(ertekek[0],
            ertekek[1], Integer.parseInt(ertekek[2]);
    }
} catch (IOException ioException) {
    ioException.printStackTrace();
}
\end{lstlisting}

\subsection{CSV fajlba iras}

A korabban tanult \l{BufferedWriter}-t hasznalo eljaras segitsegevel mi is hozhatunk letre CSV fajlokat, annyi a feladatunk, hogy az ertekeinket amiket fajlba szeretnenk irni a CSV formatum szabalyai szerint alakitjuk. Ha egy osztaly peldanyait szeretnenk CSV formaban menteni, a legelegansabb megoldas az, ha egy metodusban hatarozzuk meg, mikent reprezentaljuk majd a peldanyokat CSV-sorkent. Az \l{Auto} osztalyunkat kiegeszithetjuk az alabbi metodussal:

\begin{lstlisting}[language=Java, caption=Metodus CSV sor letrehozasahoz]
String csvSor() {
    return gyarto + "," + modell + "," + ev;
}
\end{lstlisting}

\newpage

Ezutan mindossze annyit kell tennunk, hogy az \l{autok} lista minden elemen meghivjuk ezt a metodust, es a metodus eredmenyet beleirjuk az adat.txt fajlba. Mivel a fajl elso sora eredetileg az oszlopneveket tartalmazza, nem felejtem azt is beleirni.

\begin{lstlisting}[language=Java, caption=Autok visszairasa a CSV fajlba]
try (BufferedWriter br = new BufferedWriter(
    new FileWriter(new File("adat.txt")))){
    br.write("gyarto,modell,ev\n");
    for (Auto a: autok) {
        br.write(a.csvSor());
        br.newLine();
    }
} catch (IOException ioException) {
    ioException.printStackTrace();
}
\end{lstlisting}

\newpage
\section{Feladatok}

\subsection{Feladat}

Olvasd be a feladatsorhoz mellekelt countries.txt fajlt es ird ki minden sorat a konzolra \l{FileInputStream} segitsegevel, majd tedd meg ugyanezt a \l{BufferedReader} modszerrel.

\subsection{Feladat}

Szamold meg es ki, hany orszag van osszesen a countries.txt fajlban.

\subsection{Feladat}

Ird ki a countries.txt fajlbol csak az "S" betuvel kezdodo orszagok neveit.

\subsection{Feladat}

Ird bele a countries-forditott.txt fajlba az orszagok neveit forditva (tehat az elso orszag lesz az utolso).

\subsection{Feladat}

Olvasd be a feladatsorhoz mellekelt termekek.csv fajlt egy megfeleloen osszeallitott osztalyt tartalmazo listaba. Figyeld meg alaposan, hogy milyen valtozokra lesz szukseged, milyen adattipusokat kell felhasznalnod, es ne felejtsd el atugrani az oszlopneveket tartalmazo elso sort.

\subsection{Feladat}

Az elozo feladatban beolvasott adat minden ertekenek az arat noveld 10\%-al.

\subsection{Feladat}

A modositott arakkal rendelkezo termekeket ird bele a termekek-modositott.csv fajlba.
    
\end{document}

