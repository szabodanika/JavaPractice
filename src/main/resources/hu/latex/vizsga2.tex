\documentclass{article}

\usepackage[utf8]{inputenc}
\usepackage[hungarian]{babel}
\usepackage{hyperref}
\usepackage{xcolor}
\usepackage{listings}
\renewcommand{\lstlistingname}{Kodreszlet}%
\lstset{language=Java,
backgroundcolor=\color[HTML]{ebebeb},
keywordstyle={\bfseries \color[HTML]{5d00ff}},
frame=single,
basicstyle=\footnotesize\ttfamily,
captionpos=b,
tabsize=2,
numbers=left,
aboveskip=2em,
belowskip=1em
}
\usepackage{graphicx}
\usepackage{float}
\restylefloat{table}

\title{%
Java SE Vizsga 2\\
\large Java alapok, OOP, fajlkezeles\\
\large Rendelkezesre allo ido: 90 perc}

\author{Szabo Daniel\\daniel.szabo99@outlook.com}

\date{\today}

\begin{document}

\maketitle

\section{Utasitasok}

\subsection{Projekt megnyitasa}

A vizsgahoz a kovetkezo fajlra van szukseged: fenykepezogepek.csv.

Ezeket egy uj Eclipse Java projekt-ben helyezzuk el. A projekt neve legyen \lstinline{JavaSEVizsga2_VezeteknevKeresztnev}, a projektben hozzunk letre egy \lstinline{vizsga} csomagot es abban helyezzeuk el a megoldashoz szukseges osztalyt/osztalyokat.

\subsection{Feladatok megoldasa}

\begin{itemize}
    \item A megoldas soran kovessuk az objektum orientalt programozas iranyelveit, torekedjuk osztalyokat es metodusokat hasznalni a megoldashoz.
    \item Nem mukodo kodot illetve felesleges kommenteket ne hagyjunk a projektben
\end{itemize}

\subsection{Megoldas beadasa}

A teljes projektet egy zip allomanykent mentsuk el. Ezt megtehetjuk a File $\rightarrow$ Export $\rightarrow$ Archive File menuponttal. A tomoritett allomany neve legyen a projekt nevevel azonos. Beadas elott csomagold ki a projektet es ellenorizd, hogy mukodik a megoldasod, majd add be a zip allomanyt.

\newpage

\section{Feladatok}

Olvasd be a UTF-8 kodolasu fenykepezokepek.csv fajlt, majd oldd meg a kovetkezo feladatokat. A programod egyszeri lefuttatassal minden feladatra kell szoveges valaszt adjon. Figyelmesen olvasd el az osszes feladatot, mielott hozzakezdesz a munkahoz.

\subsection{Feladat}
Tarold be a beolvasott fenykepezokepeket egy megfelelo adatszerkezetben, amit kesobb ciklusokkal be tudsz majd jarni.

\subsection{Feladat}

Ird ki a beolvasott fenykepezogepek szamat a konzolra.

\subsection{Feladat}

Ird ki az elso 5 beolvasott fenykepezogep adatait a konzolra.

\subsection{Feladat}

Ird ki az atlagos szenzor felbontast a konzolra.

\subsection{Feladat}

Ird ki minden gyartora, hogy atlagosan hany megapixel felbontasuak a fenykepezogepeik.

\subsection{Feladat}

Ird ki a legnagyobb felbontasu fenykepezogep adatait.

\subsection{Feladat}

Melyik evben jelent meg a legtobb fenykepezogep es hany darab osszesen?

\subsection{Feladat}

Hozz letre egy metodust, ami beker a felhasznalotol egy gyarto nevet, majd megjeleniti a gyarto osszes kamerajat.

\subsection{Feladat}

Hozz letre egy olyan metodust, ami beker egy evet es megjelenit minden fenykepezogepet ami abban az evben jelent meg.

\subsection{Feladat}

Hozz letre egy olyan metodust, ami bekeri egy uj fenykepezogep minden adatat es eltarolja a tobbi fenykepezogeppel egyutt, majd kiirja az osszes fenykepezogep adatait, beleertve az iment hozzaadott fenykepezot. A fenykepezogepek.csv fajlba nem kell elmentened ezt az uj adatot.


\end{document}