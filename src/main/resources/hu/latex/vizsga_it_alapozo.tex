\documentclass{article}

\usepackage[utf8]{inputenc}
\usepackage[hungarian]{babel}
\usepackage{hyperref}
\usepackage{xcolor}
\usepackage{listings}
\renewcommand{\lstlistingname}{Kodreszlet}%
\lstset{language=Java,
backgroundcolor=\color[HTML]{ebebeb},
keywordstyle={\bfseries \color[HTML]{5d00ff}},
frame=single,
basicstyle=\footnotesize\ttfamily,
captionpos=b,
tabsize=2,
numbers=left,
aboveskip=2em,
belowskip=1em
}
\usepackage{graphicx}
\usepackage{float}
\restylefloat{table}
\let\l\lstinline

\title{%
IT Szakmai Alapok Vizsga\\
\large Web Technologiak es Adatkezeles\\
\large Rendelkezesre allo ido: 90 perc}

\author{Szabo Daniel\\daniel.szabo99@outlook.com}

\date{\today}

\begin{document}

\maketitle

\section{Utasitasok}

Hozz letre egy szoveges fajlt .txt kiterjesztessel, aminek a neve legyen \\ \lstinline{ITAlapozoVizsga_VezeteknevKeresztnev.txt}. A fajl tartalmazza az elso ket sorban a mai datumot es a teljes nevedet.

Minden feladathoz ird oda, hogy hanyadik feladatra valaszolsz, majd ird le a valaszaidat minel reszletesebben es magas szakmai pontossaggal.

\subsection{Megoldas beadasa}

Amint elkeszultel a megoldasoddal, figyelmesen olvasd at megegyszer a valaszaidat majd add be a .txt fajlodat.

\newpage

\section{Feladatok}

\subsection{Feladat (5 pont)}

Fogalmazd meg egy mondatban, mire hasznaljuk a kovetkezo technologiakat:
\begin{itemize}
\item HTTPS,
\item FTP,
\item SSH,
\item Git,
\item VPN.
\end{itemize}

\subsection{Feladat  (5 pont)}

Irj egy reszletes, technikai elemzest az alabbi HTTP kerelemrol. A kerelem minden reszet jelold meg es ird le, mit csinal, osszesen nem tobb mint 6-7 mondatban.

\begin{lstlisting}[language=HTTP kerelem]
POST /vizsga/form.php HTTP/1.1
Host: pelda.hu
nev=Janos&kor=20
\end{lstlisting}

\subsection{Feladat  (4 pont)}

Fogalmazd meg a GUI ket elonyet es ket hatranyat a konzolos feluletekkel szemben.

\subsection{Feladat  (4 pont)}

Fogalmazd meg az SQL adatbazisok ket elonyet es ket hatranyat a noSQL adatbazisokkal szemben.

\subsection{Feladat (3 pont)}

Ird le minden alabbi adattarolasi formatumnak egy elonyet a masik kettovel szemben.

\begin{itemize}
\item CSV,
\item XLS,
\item adatbazis (pl. MySQL).
\end{itemize}

\newpage

\subsection{Feladat (4 pont)}

Adott a kovetkezo adattabla, aminek a neve \l{tanulok}:

\begin{table}[H]
    \begin{tabular}{|l|l|l|l|}
        \hline
        \textbf{id} & \textbf{keresztnev} & \textbf{vezeteknev} & \textbf{kor} \\ \hline
        0           & Kis                 & Janos               & 20           \\ \hline
        1           & Kozepes             & Julia               & 24           \\ \hline
        2           & Nagy                & Jolan               & 27           \\ \hline
        3           & Kis                 & Jakab               & 24           \\ \hline
    \end{tabular}
\end{table}

Mi lesz a kovetkezo SQL lekerdezes eredmenye? A valaszodat indokold is.

\begin{lstlisting}[language=SQL lekerdezes]
SELECT keresztnev, kor
FROM tanulok
WHERE vezeteknev = 'Kis'
ORDER BY kor;
\end{lstlisting}

\end{document}