\documentclass{article}

\usepackage[utf8]{inputenc}
\usepackage[hungarian]{babel}
\usepackage{hyperref}
\usepackage{xcolor}
\usepackage{listings}
\renewcommand{\lstlistingname}{Kodreszlet}%
\lstset{language=Java,
backgroundcolor=\color[HTML]{ebebeb},
keywordstyle={\bfseries \color[HTML]{5d00ff}},
frame=single,
basicstyle=\footnotesize\ttfamily,
captionpos=b,
tabsize=2,
numbers=left,
aboveskip=2em,
belowskip=1em
}
\usepackage{graphicx}
\let\l\lstinline

\title{%
Java SE Feladatsor 2 \\
\large Adatfeldolgozas}

\author{Szabo Daniel\\daniel.szabo99@outlook.com}

\date{\today}

\begin{document}

\maketitle
\begin{abstract}
Ebben a feladatsorban fajlokbol beolvasott, CSV formatumu adattal fogunk dolgozni es komplex statisztikakat szamolni. A feladatsor sikeres megoldasa kritikus fontossagu a kesobbi vizsgakhoz, mert alapveto fajlmuveleteket es szamitasokat erint, mint atlagszamitas, csoportositas, szures, extrem ertek kereses illetve ezek kombinacioi.
\end{abstract}

\newpage

\tableofcontents{}

\newpage

\section{Feladatok}

\paragraph{Bevezetes}

Ebben a feladatsorban egy konyvesbolt adatait fogjuk beolvasni, majd a beolvasott adatokon muveleteket vegezni, statisztikakat szamolni. A feladatsorhoz mellekelt kovetkezo fajlokra lesz szukseged: \newline
\begin{itemize}
    \item konyvek.csv,
    \item szerzok.csv,
    \item kiadok.csv,
    \item Main.java,
    \item Feladatok.java.
\end{itemize}


\subsection{Feladat}
Figyelmesen olvasd at a mellekelt csv fajlokat es azoknak megfeleloen hozz letre modell osztalyokat, amik a fajlok egy-egy sorat hianytalanul eltarolni kepesek.

\subsection{Feladat}
A Feladatok osztaly \l{feladat2()} metodusat egeszitsd ki ugy, hogy olvassa be a harom csv fajl adatait, helyezze a sorokat objektumokba es tarolja el azokat a megfelelo listakban.

\subsection{Feladat}
A Feladatok osztaly \l{feladat3()} metodusat egeszitsd ki ugy, hogy jelenitse meg mindharom beolvasott adattipus listajanak elso 2 elemet.

\subsection{Feladat}
A Feladatok osztaly \l{feladat4()} metodusat egeszitsd ki ugy, hogy szamolja ki es jelenitse meg a konyvek atlagos arat forintban, 2 tizedes pontossaggal.

\subsection{Feladat}
A Feladatok osztaly \l{feladat5()} metodusat egeszitsd ki ugy, hogy jelenitse meg a szerzo nevet akinek a legtobb konyve van a listankon, es hogy osszesen hany darab.

\subsection{Feladat}
A Feladatok osztaly \l{feladat6()} metodusat egeszitsd ki ugy, hogy keresse meg es jelenitse meg a legdragabb konyv cimet es arat.

\subsection{Feladat}
A Feladatok osztaly \l{feladat7()} metodusat egeszitsd ki ugy, hogy listazza novekvo sorrendben az osszes evet amiben lett konyv kiadva.

\subsection{Feladat}
A Feladatok osztaly \l{feladat8()} metodusat egeszitsd ki ugy, hogy az 1990 utan (tehat 1991 es kesobb) szuletett szerzok altal irt konyvek atlagarat szamitsa ki es jelenitse meg.

\subsection{Feladat}
A Feladatok osztaly \l{feladat9()} metodusat egeszitsd ki ugy, hogy keresse, majd jelenitse meg a konyvkiadot aki a legtobb konyvet adta ki 1990 utan szuletett szerzoktol, es hany darabot osszesen.

\subsection{Feladat}
A Feladatok osztaly \l{feladat10()} metodusat egeszitsd ki ugy, hogy szamitsa ki, melyik kiado konyvei kerulnek a legtobbe forint/oldal alapon, majd jelenitse meg a kiado nevet es forint/oldal atlagat.

\newpage

\section{Eredmenyek}

Ha sikeresen megoldottad a feladatokat, a kovetkezo eredmenyt kell lasd. Aprobb formai elteresek megengedettek.

\begin{lstlisting}[language=Java, caption=Kimenet a feladatok megoldasa utan]
###		FELADAT 2		###
Beolvasas Sikeres!
###		FELADAT 3		###
Szerzo{szam=1, nev='Kis Julia', szuletesiEv=1983}
Szerzo{szam=2, nev='Nagy Janos', szuletesiEv=1962}
Kiado{szam=1, nev='Oracle Konyvkiado', szekhely='Debrecen'}
Kiado{szam=2, nev='Eclipse Konyvkiado', szekhely='Budapest'}
Konyv{szam=1, cim='Java SE Programozas 1', szerzo=Szerzo{szam=1,
            nev='Kis Julia', szuletesiEv=1983}, kiado=Kiado{szam=1,
            nev='Oracle Konyvkiado', szekhely='Debrecen'}, ev=2019,
            oldal=213, ar=1999, kemenyboritos=true}
Konyv{szam=2, cim='Java SE Programozas 2', szerzo=Szerzo{szam=1,
            nev='Kis Julia', szuletesiEv=1983}, kiado=Kiado{szam=1,
            nev='Oracle Konyvkiado', szekhely='Debrecen'}, ev=2020,
            oldal=206, ar=1999, kemenyboritos=true}
###		FELADAT 4		###
Atlagos konyv ar: 3061.40 Ft
###		FELADAT 5		###
A legtobb konyve Kis Julia szerzonek van: 8 db
###		FELADAT 6		###
A legdragabb konyv A Webfejlesztes Enciklopediaja - 7899 Ft
###		FELADAT 7		###
2009
2011
2012
2014
2016
2017
2019
2020
2021
###		FELADAT 8		###
1990 utan szuletett szerzok konyveinek atlagos ara: 3760.25 Ft
###		FELADAT 9		###
A legtobb 1990 utani szuletesu szerzo altal
    irt konyvet Oracle Konyvkiado adta ki: 3 db
###		FELADAT 10		###
A legnagyobb atlagos Ft/oldal erteku kiado
    Konstans Tankonyvkiado: 15.02 Ft
\end{lstlisting}

\newpage

\section{Megoldas beadasa}
Ha minden feladatot megoldottal es az osszes teszt sikeresen lefut, akkor gyozodj meg rola, hogy a kodod tiszta es olvashato, nem tartalmaz felesleges, kommentelt vagy nem mukodo kodot illetve minden feladat megoldasahoz objektum orientalt programozast alkalmaztal amennyire csak lehetseges volt, majd a teljes projektet (tehat nem csak a src mappat vagy csak a .java fajlokat) csomagold be egy .zip kiterjesztesu fajlba, aminek a neve legyen peldaul \lstinline{KisJuliska_JavaSE2.zip}. Ezt a fajlt utana vagy el kell kuldened e-mail-ben vagy fel kell feltoltened, errol kulon fogsz pontos utmutatast kapni.

\end{document}